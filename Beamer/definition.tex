\section{Problemdefinition und wichtige Größen}

\begin{frame}
	\renewcommand{\thisthmnumber}{}
	\begin{definition}[Gemischt-ganzzahliges lineares Programm]
		Für $A\in\Z^{n\times m}$, $b\in\Z^m$, $c\in\R^n$ und $I\subseteq\firstNumbers{n}:=\{1,\dots,n\}$ bezeichne ($I$-\MIPI) das Programm
		$$\begin{array}{lc}
		&\max c\transpose x \\
		\subjectTo &Ax\leq b\\
		&\forall i\in I: x_i\in\Z
		\end{array}.$$
	\end{definition}

	\pause
	Mit $\Delta(A):=\max\{\betrag{ \det(Q)} \mid Q \text{ quadratische Untermatrix von } A \}$ formulieren wir folgende Vermutung:
	
	\pause
	\renewcommand{\thisthmnumber}{1.1}
	\begin{conjecture}
		Es gibt eine Funktion $f: \N\rightarrow\R$, sodass für alle $I,J\subseteq\firstNumbers{n}$ gilt:
		
		Besitzt ($J$-\MIPI) eine Optimallösung, so existiert für jede Optimallösung $x^*$ von ($I$-\MIPI) eine Optimallösung $y^*$ von ($J$-\MIPI) mit $\norm{x^* - y^*}\leq f(\Delta)$.
	\end{conjecture}
\end{frame}

\subsection{Einfache Folgerung aus Theorem von Cook}
\begin{frame}{Theorem von Cook}
\renewcommand{\thisthmnumber}{2.1}
\begin{thm}[Cook u. a., 1986]\label{thm:cook}
	Seien $I, J\subseteq\firstNumbers{n}$ mit $I=\emptyset$ oder $J=\emptyset$, sodass ($J$-\MIPI) eine Optimallösung hat.
	
	Dann existiert für jede Optimallösung $x^*$ von ($I$-\MIPI) eine Optimallösung $y^*$ von ($J$-\MIPI) mit $\norm{x^*-y^*}\leq n\Delta$.
\end{thm}
\pause
\renewcommand{\thisthmnumber}{2.2}
\begin{korollar}
	Seien $I,J\subseteq\firstNumbers{n}$, sodass ($J$-\MIPI) eine Optimallösung hat.
	
	Dann existiert für jede Optimallösung $x^*$ von ($I$-\MIPI) eine Optimallösung $y^*$ von ($J$-\MIPI) mit $\norm{x^*-y^*}\leq2 n\Delta$.
\end{korollar}
\pause
Wir wollen in dieser Abschätzung $2n$ durch $\betrag{I\cup J}$ ersetzen.
\end{frame}

