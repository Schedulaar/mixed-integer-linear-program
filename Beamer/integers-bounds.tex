\section{Abschätzung mit Anzahl ganzzahliger Variablen}

\subsection{Beweis der Abschätzung}

\begin{frame}{Hilfslemmata und Theorem}
	\begin{lemma}
		Sei $A\in\Z^{m\times n}$ eine Matrix, deren Zeilen in zwei Untermatrizen $A_1$ und $A_2$ aufgeteilt sind.
		
		Der Polyeder $C:=\{ x\in\R^n \mid A_1x\leq 0, A_2x\geq0 \}$ besitzt die Darstellung $C=\{\sum_{i=1}^k\lambda_i v^i \mid \forall i\in\firstNumbers{k} \colon\lambda_i\geq0 \}$ mit $v^i\in\Z^n$ und $\norm{v^i}\leq\Delta(A)$ für $i\in\firstNumbers{k}$.
	\end{lemma}
	\pause
	\begin{lemma}\label{lem:maxgamma}
		Seien $u^i\in\Z^d$, $\lambda_i\geq0$ für $i\in\firstNumbers{k}$ gegeben.
		Dann existiert $\gamma\in\bigtimes_{i=1}^k [0,\lambda_i]$ mit $\sum_{i=1}^k \gamma_i u^i\in\Z^d$ und  $\sum_{i=1}^k(\lambda_i-\gamma_i)<d$.
	\end{lemma}
	\pause
	\begin{theorem}\label{thm:theo2}
		Seien $I,J\subseteq\firstNumbers{n}$, sodass ($J$-\MIPI) eine Optimallösung $\tilde{y}$ hat.
		
		Für jede Optimallösung $x^*$ von ($I$-\MIPI) existiert eine Optimallösung $y^*$ von ($J$-\MIPI) mit $\norm{x^*-y^*}\leq\betrag{I\cup J}\cdot\Delta$.
	\end{theorem}
\end{frame}

\begin{frame}{Hilfslemmata und Theorem}
\begin{theorem}\label{thm:theo2}
Seien $I,J\subseteq\firstNumbers{n}$, sodass ($J$-\MIPI) eine Optimallösung $\tilde{y}$ hat.

Für jede Optimallösung $x^*$ von ($I$-\MIPI) existiert eine Optimallösung $y^*$ von ($J$-\MIPI) mit $\norm{x^*-y^*}\leq\betrag{I\cup J}\cdot\Delta$.
\end{theorem}
Zur Zulässigkeit von $y^*$ für ($J$-\MIPI) und $\tilde{x}$ für ($I$-\MIPI):
\pause
$$
\begin{array}{lllllll}
A_1 y^*&=&A_1x^*+\sum_{i=1}^k(\lambda_i-\gamma_i)A_1v^i &\leq& A_1 x^* &\leq&b_1\\ \pause
A_2 y^*&=&A_2\tilde{y}-\sum_{i=1}^k\gamma_iA_2v^i &\leq& A_2\tilde{y} &\leq& b_2\\ \pause
A_1\tilde{x} &=&A_1x^*+\sum_{i=1}^k \gamma_iA_1v^i &\leq& A_1x^* &\leq&b_1\\ \pause
A_2\tilde{x} &=&A_2\tilde{y}-\sum_{i=1}^k(\lambda_i-\gamma_i)A_2v^i&\leq& A_2\tilde{y}&\leq& b_2.
\end{array}
$$
\end{frame}

\subsection{Folgerungen aus Theorie endlicher Gruppen}

\begin{frame}{Das Theorem von Olson}

	\begin{definition}[Davenport-Konstante]
		Sei $(G,+,0)$ eine endliche, abelsche Gruppe.
		Die {\em Davenport-Konstante $D(G)$} ist die kleinste Zahl $k\in\N$ mit 
		$$
		\forall g^1,\dots,g^k \in G~~\exists I\subseteq\firstNumbers{k}\colon I\neq\emptyset \wedge \sum_{i\in I}g^i=0 .
		$$
	\end{definition}

	\begin{theorem}[Olson, 1969]
		Die Davenport-Konstante von $\Z^d/p\Z^d$ ist $1+dp-d$ für $p$ prim.
	\end{theorem}

	\pause
	\begin{korollar}
		Seien $p\in\N$ eine Primzahl, $d\in\N$ und $f^1,\dots,f^r\in\Z^d$ mit $r\geq 1+dp-d$ gegeben.
		
		Dann existiert eine nicht-leere Menge $I\subseteq\firstNumbers{r}$ mit $\sum_{i\in I}f^i\in p\Z^d$.
	\end{korollar}
\end{frame}