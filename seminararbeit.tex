\documentclass[paper=a4, 	% Seitenformat
		fontsize=11pt, 		% Schriftgr\"o\ss{}e
		abstracton, 	% mit Abstrakt
		headsepline, 	% Trennlinie f\"ur die Kopfzeile
		notitlepage	% keine extra Titelseite
		]{scrartcl}

\usepackage[utf8]{inputenc}

%%%%%%%%%%%%%%%%%%%%%%%%%%%%%%%%%%%%%%%%%%%%%%%%%%%%%%%%%%%%%%%%%%%%%%%%%%%%%%%%%%%%%%%%%%%
% Zusammenfassung einiger nützlicher Pakete und Befehle
%-------------------------------------------------------------------------------
% Kopf-Zeilen
%-------------------------------------------------------------------------------

\usepackage[automark]{scrpage2}	% Seiten-Stil f\"ur scrartcl
\pagestyle{scrheadings}		% Kopfzeilen nach scr-Standard		
\ifx\chapter\undefined 		% falls Kapitel nicht definiert sind
  \automark[subsection]{section}% Kopf- und Fusszeilen setzen
\else				% Kapitel sind definiert
  \automark[section]{chapter}	% Kopf- und Fusszeilen setzen
\fi

%-------------------------------------------------------------------------------
%   Maske f\"ur \"Uberschrift 
%-------------------------------------------------------------------------------
% Belegung der notwendigen Kommandos f\"ur die Titelseite
\newcommand{\autor}{Markl, Michael} 		% bearbeitender Student
\newcommand{\veranstaltung}{Seminar zur Optimierung und Spieltheorie} 	% Titel des ganzen Seminars
\newcommand{\uni}{Institut für Mathematik der Universität Augsburg} % Universit\"at
\newcommand{\lehrstuhl}{Diskrete Mathematik, Optimierung und Operations Research} % Lehrstuhl
\newcommand{\semester}{Wintersemester 2018/19}	% Winter- oder Sommersemester mit Jahr
\newcommand{\datum}{08.11.2018} 			% Datumsangabe
\newcommand{\thema}{Abstände optimaler Lösungen gemischt-ganzzahliger Programme}  		% Titel der Seminararbeit

\newcommand{\ownline}{\vspace{.7em}\hrule\vspace{.7em}} % horizontale Linie mit Abstand

\newcommand{\seminarkopf}{	% Befehl zum Erzeugen der Titelseite 
 \textsc{\autor}  \hfill{\datum} \\ 
\textbf{\veranstaltung} \\ 
\uni \\ 
\lehrstuhl \\
\semester 
\ownline 

\begin{center}
{\LARGE \textbf{\thema}}
\end{center}

\ownline
}			    % Befehle und Pakete für Titelseite
% Mathematische Zeichens\"atze und Umgebungen
\usepackage{amsfonts, amssymb}	% Definition einer Liste mathematischer Fontbefehle und Symbole
\usepackage[intlimits,		% Integralgrenzen \"uber und unter dem Integral
	    sumlimits]		% Summationsgrenzen \"uber und unter der Summe
           {amsmath}		% mathematische Verbesserungen
\usepackage{amsthm}		% spezielle theorem Stile
\usepackage{aliascnt} 

%-------------------------------------------------------------------------------
% Hilfreiche Befehle
%-------------------------------------------------------------------------------
\newcommand{\betrag}[1]{\lvert #1 \rvert}	        % Betrag
\providecommand*{\Lfloor}{\left\lfloor}                 % gro\ss{}es Abrunden
\providecommand*{\Rfloor}{\right\rfloor}                % gro\ss{}es Abrunden
\providecommand*{\Floor}[1]{\Lfloor #1 \Rfloor}         % gro\ss{}es ganzes Abrunden
\providecommand*{\Ceil}[1]{\left\lceil #1 \right\rceil} % gro\ss{}es ganzes Aufrunden

\DeclareMathOperator{\e}{ex}
\DeclareMathOperator{\ma}{mate}
\DeclareMathOperator{\Ex}{Ex}

%-------------------------------------------------------------------------------
%   Befehle für Nummerierung der Ergebnisse
%   fortlaufend innerhalb eines Abschnittes
%-------------------------------------------------------------------------------
\theoremstyle{plain}            % normaler Stil
\newtheorem{theorem}{Theorem}[section]
% Lemma
\newaliascnt{lemma}{theorem}
\newtheorem{lemma}[lemma]{Lemma}
\aliascntresetthe{lemma}
% Satz
\newaliascnt{satz}{theorem}
\newtheorem{satz}[satz]{Satz}
\aliascntresetthe{satz}
% Korollar
\newaliascnt{corollary}{theorem}
\newtheorem{corollary}[corollary]{Korollar}
\aliascntresetthe{corollary}
% Proposition
\newaliascnt{proposition}{theorem}
\newtheorem{proposition}[proposition]{Proposition}
\aliascntresetthe{proposition}
%-------------------------------------------------------------------------------
\theoremstyle{definition}	% Definitionsstil
% Definition
\newaliascnt{definition}{theorem}
\newtheorem{definition}[definition]{Definition}
\aliascntresetthe{definition}
% Beispiel
\newaliascnt{beispiel} {theorem}
\newtheorem{beispiel}[beispiel]{Beispiel}
\aliascntresetthe{beispiel}
% Problem
\newaliascnt{problem}{theorem}
\newtheorem{problem}[problem]{Problem}
\aliascntresetthe{problem}
% Algorithmus
\newaliascnt{algorithmus}{theorem}
\newtheorem{algorithmus}[algorithmus]{Algorithmus}
\aliascntresetthe{algorithmus}
%-------------------------------------------------------------------------------
\theoremstyle{remark}		% Bemerkungsstil
% Bemerkung
\newaliascnt{bemerkung}{theorem}
\newtheorem{bemerkung}[bemerkung]{Bemerkung}
\aliascntresetthe{bemerkung}
% Vermutung
\newaliascnt{conjecture}{theorem}
\newtheorem{conjecture}[conjecture]{Vermutung}
\aliascntresetthe{conjecture}
% Notation
\newaliascnt{notation}{theorem}
\newtheorem{notation}[notation]{Notation}
\aliascntresetthe{notation}

%-------------------------------------------------------------------------------
% automatische Referenzen mit interaktiven Text
%-------------------------------------------------------------------------------

% Texte
\newcommand{\theoremautorefname}{Theorem}
\newcommand{\lemmaautorefname}{Lemma}
\newcommand{\satzautorefname}{Satz}
\newcommand{\korollarautorefname}{Korollar}
\newcommand{\propositionautorefname}{Proposition}

\newcommand{\definitionautorefname}{Definition}
\newcommand{\beispielautorefname}{Beispiel}
\newcommand{\problemautorefname}{Problem}
\newcommand{\algorithmusautorefname}{Algorithmus}

\newcommand{\bemerkungautorefname}{Bemerkung}
\newcommand{\vermutungautorefname}{Vermutung}
\newcommand{\notationautorefname}{Notation}

%-------------------------------------------------------------------------------
% Nummerierung der Gleichungen innerhalb der obersten Ebene
%-------------------------------------------------------------------------------
\ifx\chapter\undefined 			% Kapitel sind definiert
  \numberwithin{equation}{section}	% Gleichungsnummern in Section
\else					% Kapitel sind nicht definiert
  \numberwithin{equation}{chapter}	% Gleichungsnummern in Kapiteln
\fi
			% Mathematische Befehle und Pakete

% Literatur-Bibliothek
\bibliographystyle{alphadin}               % deutscher Bibliotheksstil

% Interaktive Referenzen, und PDF-Keys
\usepackage{xr-hyper}  
\usepackage[pagebackref,                % R\"uckreferenz im Literaturverzeichnis
           pdftex,                      % Treiber f\"ur ps zu pdf ; f\"ur direkt nach pdf: pdftex
           ]{hyperref}

% Erweiterte Einstellungen zu hyperref

\hypersetup{
        breaklinks=true,        % zu lange Links unterbrechen
        colorlinks=true,        % F\"arben von Referenzen
        citecolor=black,        % Farbe der Zitate
        linkcolor=black,        % Farbe der Links
        extension=pdf,          % Externe Dokumente k\"onnen eingebunden werden.
        ngerman,		
	pdfview=FitH,
	pdfstartview=FitH,		
	bookmarksnumbered=true, % Anzeige der Abschnittsnummern	% pdf-Titel
	pdfauthor={\autor}% pdf-Autor
}

% Namen f\"ur Referenzen 

\newcommand{\ownautorefnames}{
  \renewcommand{\sectionautorefname}{Kapitel}
  \renewcommand{\subsectionautorefname}{Unterkapitel}
  \renewcommand{\subsubsectionautorefname}{\subsectionautorefname}
  \renewcommand{\appendixautorefname}{Anhang}
  \renewcommand{\figureautorefname}{Abbildung}
}

% R\"uckreferenzentext zur Literatur
\def\bibandname{und}%
\renewcommand*{\backref}[1]{}
\renewcommand*{\backrefalt}[4]{%
\ifcase #1 %
 (Nicht zitiert, also Erg\"anzungsliteratur.)%
\or
 (Zitiert auf Seite #2.)%
\else
 (Zitiert auf den Seiten #2.)%
\fi
}
\renewcommand{\backreftwosep}{ und~} % seperate 2 pages
\renewcommand{\backreflastsep}{ und~} % seperate last of longer 

			% Befehle und Pakete für Referenzen
\usepackage{array}		% erweiterte Tabellen

% Schriftzeichen, Format
\usepackage{latexsym}		% Latex-Symbole
\usepackage[latin1]{inputenc}	% Eingabekodierungen
\usepackage[english,ngerman]{babel}	% Mehrsprachenumgebung

% Layout
\usepackage{geometry}                    % Seitenränder
\geometry{a4paper, top=30mm, bottom=30mm, left=30mm, right=30mm}
\addtolength{\footskip}{-0.5cm}          % Seitenzahlen höher setzen
\usepackage{xcolor}                      % Farben


% Tabellen und Listen
\usepackage{float}		        % Gleitobjekte 
\usepackage[flushright]{paralist}       % Bessere Behandlung der Auflistungen

% Bilder
\usepackage[final]{graphicx}            % Graphiken einbinden

\usepackage{caption}                    % Beschriftungen
\usepackage{subcaption}                 % Beschriftungen f\"ur Unterteilung

\usepackage{pst-all}                    % Zeichnungen in Latex (kein pdflatex)
\usepackage{pstricks-add}               % zus\"atzliches von pstricks
\usepackage{pst-3dplot}                 % dreidimensionale Zeichnungen
\usepackage{pst-eucl}                   % euklidisches Paket

\numberwithin{figure}{section}	% Abbildungsnummern in Section			    % restliche Befehle und Pakete

%%%%%%%%%%%%%%%%%%%%%%%%%%%%%%%%%%%%%%%%%%%%%%%%%%%%%%%%%%%%%%%%%%%%%%%%%%%%%%%%
%%%%%%%%%%%%%%%%%%%%%%%%%%%%%%%%%%%%%%%%%%%%%%%%%%%%%%%%%%%%%%%%%%%%%%%%%%%%%%%%
% Start des Dokuments
\begin{document}		

\ownautorefnames		% Änderung einiger automatischen Texte von hyperref (wie in referenz.tex definiert)
\parindent0em 			% kein Einzug nach einer Leerzeile

%%%%%%%%%%%%%%%%%%%%%%%%%%%%%%%%%%%%%%%%%%%%%%%%%%%%%%%%%%%%%%%%%%%%%%%%%%%%%%%%
% Titelseite
\thispagestyle{empty}		% leerer Seitenstil, also keine Seitennummer
\begin{titlepage}
\seminarkopf 			% Titelblatt (wie in kopf.tex definiert)
\begin{abstract} 
Ein gemischt-ganzzahliges Programm ist ein lineares Optimierungsprogramm, bei dem die Variablen einer bestimmten Indexmenge auf die ganzen Zahlen beschränkt sind.
Wir verstärken bisherige Resultate, die eine Abschätzung optimaler Lösungen zweier gemischt-ganzzahliger Programme, die sich nur in der Indexmenge unterscheiden, anhand der Anzahl an Variablen und $\Delta$ geben.
Die Größe $\Delta$ quantifiziert dabei den größten Absolutwert der Determinanten aller quadratischer Untermatrizen.
Wir zeigen eine Abschätzung, die nur die beiden Indexmengen und $\Delta$ verwendet, und vermuten, dass der Abstand gemischt-ganzzahliger Probleme sogar linear von $\Delta$ abhängt, wobei wir Szenarien untersuchen, die diese Vermutung bestätigen.
\end{abstract} 
\end{titlepage}

%%%%%%%%%%%%%%%%%%%%%%%%%%%%%%%%%%%%%%%%%%%%%%%%%%%%%%%%%%%%%%%%%%%%%%%%%%%%%%%%
% Inhaltsverzeichnis
\thispagestyle{empty}	
\tableofcontents		% Inhaltsverzeichnis
%\listoffigures			% Abbildungsverzeichnis (eventuell einfügen)
%\listoftables			% Tabellenverzeichnis (eventuell einfügen)
\setcounter{page}{0}% Eigentlicher Inhalt beginnt auf Seite 1
\clearpage          % neue Seite für eigentlichen Inhalt
%%%%%%%%%%%%%%%%%%%%%%%%%%%%%%%%%%%%%%%%%%%%%%%%%%%%%%%%%%%%%%%%%%%%%%%%%%%%%%%%
% Eigentlicher Inhalt der Seminararbeit; die einzelnen Teile werden hier (aus Gründen der Übersichtlichkeit) über \input{file} eingebunden

\section{Gemischt-ganzzahlige lineare Programme}\label{introduction}

Diese Arbeit behandelt die Abschätzung von Abständen optimaler Lösungen
gemischt-ganzzahliger Programme.
Ein gemischt-ganzzahliges Programm ist ein lineares Programm, bei dem die
Matrix, die rechte Seite $b$ sowie einige Variablen ganzzahlig und die
restlichen Variablen reell sind.
Speziell erzeugen $A\in\Z^{n\times m}$, $b\in\Z^m$, $c\in\R^n$ und $I\subseteq\firstNumbers{n}:=\{1,\dots,n\}$ das gemischt-ganzzahlige Programm
\begin{equation}\tag{$I$-\MIPI}
\begin{array}{lc}
	&\max c\transpose x \\
	\subjectTo &Ax\leq b\\
	&\forall i\in I: x_i\in\Z
\end{array}.
\end{equation}

Wir schreiben ($I$-\MIPR), falls zusätzlich $b\in\R^m$ zugelassen wird.
Ist also beispielsweise $I=\emptyset$, erhält man ein Problem in Standardform; $I=\firstNumbers{n}$ bildet ein rein ganzzahliges lineares Programm.

Gesucht ist nun für $I,J\subseteq\firstNumbers{n}$ eine möglichst kleine Schranke, die den Abstand zwischen jeder optimalen Lösung $x^*$ von ($I$-\MIPI) und einer nähesten optimalen Lösung $y^*$ von ($J$-\MIPI) beschränkt.
Diese Schranke soll außerdem nur von $A$, $I$ und $J$ abhängen und der Abstand mittels Maximumsnorm ermittelt werden.
Desweiteren wird immer vorausgesetzt, dass ($J$-\MIPI) eine optimale Lösung besitzt.
Definiert man
$$\Delta:=\Delta(A):=\max\{\betrag{ \det(Q)} \mid Q \text{ quadratische Untermatrix von } A \},$$
so erhält man die folgende in \cite{Paat2018} formulierte Vermutung.

\begin{conjecture}\label{con:delta}
	Es gibt eine Funktion $f: \N\rightarrow\R$, sodass für alle $I,J\subseteq\firstNumbers{n}$, unter denen ($J$-\MIPI) eine optimale Lösung besitzt, gilt:
	Besitzt ($I$-\MIPI) eine optimale Lösung $x^*$, so existiert eine optimale 
	Lösung $y^*$ von ($J$-\MIPI) mit $\norm{x^* - y^*}\leq f(\Delta)$.
\end{conjecture}

In dieser Arbeit werden wir eine etwas schwächere Aussage in Theorem~\ref{thm:theo2} zeigen, bei der der Abstand zusätzlich noch von $\betrag{I\cup J}$ abhängt.
Des Weiteren werden wir in Abschnitt~\ref{sec:linear} Fälle diskutieren, in denen wir sogar die verstärkte Vermutung der Linearität von $f$ beweisen können.


\section{Abschätzungen mit Anzahl ganzzahliger Variablen}

Um eine simple Abschätzung, die zusätzlich von der Dimension $n$ abhängt, herzuleiten, nutzen wir das folgende Theorem, das von Cook u. a. in~\cite[Theorem 1 und Bemerkung 1]{Cook1986} formuliert wurde:

\begin{theorem}[Cook u. a., 1986]\label{thm:cook}
	Seien $I, J\subseteq\firstNumbers{n}$, sodass ($J$-\MIPR) eine optimale Lösung hat und entweder $I=\emptyset$ oder $J=\emptyset$ gilt.
	Dann existiert für jede optimale Lösung $x^*$ von ($I$-\MIPR) eine optimale Lösung $y^*$ von ($J$-\MIPR) mit $\norm{x^*-y^*}\leq n\Delta$.
\end{theorem}

Mit Hilfe dieses Theorems können wir nun für allgemeine Indexmengen eine ähnliche obere Grenze finden:
\begin{corollary}
	Seien $I,J\subseteq\firstNumbers{n}$, sodass ($J$-\MIPR) eine optimale Lösung
	hat.
	Dann existiert für jede optimale Lösung $x^*$ von ($I$-\MIPR) eine optimale Lösung $y^*$ von ($J$-\MIPR) mit $\norm{x^*-y^*}\leq2 n\Delta$.
\end{corollary}
\begin{proof}
	Sei $x^*$ optimale Lösung von ($I$-\MIPR).
	Nach Theorem~\ref{thm:cook} existiert eine optimale Lösung $z^*$ von ($\emptyset$-\MIPR) mit $\norm{x^*-z^*}\leq n\Delta$ und eine optimale Lösung $y^*$ von \mbox{($J$-\MIPR)} mit $\norm{z^*-y^*}\leq n\Delta$.
	
	Nach Dreiecksungleichung ist $\norm{x^*-y^*}\leq\norm{x^*-z^*}+\norm{z^*-y^*}\leq 2 n\Delta$.
\end{proof}

In diesem Abschnitt wollen wir nun diese Aussage verstärken, indem wir in der Abschätzung $2 n$ durch $\betrag{I\cup J}$ ersetzen, also durch die Anzahl der Variablen, die in ($I$-\MIPR) oder \mbox{($J$-\MIPR)} ganzzahlig sind.

\subsection{Folgerungen aus Davenport-Konstante von $p$-Gruppen}

Ein wichtiges Hilfslemma dafür lässt sich aus der sogenannten Davenport-Konstante herleiten:

\begin{definition}[Davenport-Konstante]
	Sei $(G,+,0)$ eine endliche, abelsche Gruppe.
	Die {\em Davenport-Konstante} von $G$ ist
	$$
		D(G):=\min\{k\in\N \mid \forall g^1,\dots,g^k \in G~\exists I\subseteq\firstNumbers{k}\colon I\neq\emptyset \wedge \sum_{i\in I}g^i=0  \} .
	$$
\end{definition}

Olson hat in~\cite{Olson1969} die Davenport-Konstante für sogenannte $p$-Gruppen ermittelt:
\begin{definition}[$p$-Gruppe]
	Sei $p$ eine Primzahl.
	Eine $p$-Gruppe $G$ ist eine Gruppe, in der die Ordnung jedes Elements eine Potenz von $p$ ist.
\end{definition}
\begin{theorem}[Hauptsatz endlicher abelscher Gruppen]
	Sei $G$ eine endliche abelsche Gruppe.
	$C_n$ bezeichne die zyklische Gruppe mit Ordnung $n$.
	Dann existiert eine eindeutige Darstellung $G\cong C_{n_1}\times\cdots\times C_{n_d}$ mit $2\leq n_1\mid\cdots\mid n_d$.
	Die Zahlen $n_1,\dots,n_d$ werden dabei als die sogenannten \emph{invarianten Faktoren von $G$} bezeichnet.
\end{theorem}

\begin{theorem}[Olson]\label{thm:olson}
	Für eine endliche abelsche $p$-Gruppe $G$ mit invarianten Faktoren $p^{e_1},\dots,p^{e_r}$ ist die Davenport-Konstante $D(G)=1+\sum_{i=1}^r(p^{e_i}-1)$.
\end{theorem}

Mit diesem Ergebnis können wir leicht eine für uns relevante Folgerung beschreiben:

\begin{corollary}\label{cor:olson}
	Seien $d\in\N$ und $p\in\N$ eine Primzahl sowie $f^1,\dots,f^r\in\Z^d$ mit $r\geq 1+dp-d$.
	Dann existiert eine nicht-leere Menge $I\subseteq\firstNumbers{r}$ mit $\sum_{i\in I}f^i\in p\Z^d$.
\end{corollary}
\begin{proof}
	Die $d$ invarianten Faktoren der $p$-Gruppe $\Z^d/p\Z^d\cong C_p\times\dots\times C_p$ sind jeweils $p$ und mit Theorem~\ref{thm:olson} ist $D(\Z^d/p\Z^d)=1+\sum_{i=1}^d(p-1)=1+dp-d$.
	
	Nach Definition der Davenport-Konstante existiert eine nichtleere Menge $I\subseteq\firstNumbers{1+dp-d}$ mit $\sum_{i\in I}[f^i]_p=[0]_p=p\Z^d$.
	Mit $\firstNumbers{1+dp-d}\subseteq\firstNumbers{r}$ und $\sum_{i\in I}[f^i]_p=[\sum_{i\in I}f^i]_p$ folgt die Behauptung.
\end{proof}

Damit können wir das folgende Lemma zeigen, das wir im nächsten Abschnitt benötigen werden:

\begin{lemma}\label{lem:olson}
	Seien $d,k\in\N, u^1,\dots, u^k\in\Z^d$ und $\alpha_1,\dots,\alpha_k\geq0$ mit $\sum_{i=1}^k \alpha_i\geq d$.
	Dann existiert $\beta\in\bigtimes_{i=1}^k[0,\alpha_i]$ mit $\beta\neq0$ und $\sum_{i=1}^k\beta_i u^i \in\Z^d$.
\end{lemma}
\begin{proof}
	\newcommand{\bbeta}{\tilde{\beta}}
	Ohne Beschränkung der Allgemeinheit sei $\alpha_i>0$ für $i=1,\dots,k$, denn für $\alpha_i=0$ wird $\beta_i=0$ vorausgesetzt, wodurch das Resultat nicht verändert werden kann.
	
	Zunächst betrachten wir den Fall der Existenz einer Primzahl $p$, sodass   $\alpha_i=q_i / p$ mit $q_i\in\N$ für $i\in\firstNumbers{k}$.
	Wir können nun Korollar~\ref{cor:olson} auf die Vektoren
	$$\underbrace{u^1,\dots,u^1}_{q_1~\text{Einträge}},~\dots~,\underbrace{u^k,\dots,u^k}_{q_k~\text{Einträge}}$$
	anwenden, da $r:=\sum_{i=1}^k q_i=(\sum_{i=1}^k \alpha_i)\cdot p\geq dp \geq 1+dp-d$ nach Annahme.
	Dadurch erhalten wir $l_i\in\{0,\dots,q_i\}$ für $i\in\firstNumbers{k}$ mit nicht alle $l_i=0$ und $\sum_{i=1}^k l_i u^i\in p\Z^d$.
	Teilen wir durch $p$ gelangen wir mit $\beta_i := l_i/p$ zu unserer Behauptung $\sum_{i=1}^k \beta_i u^i\in\Z^d$ und $\beta\neq0$ sowie $\beta_i\in[0,\alpha]$, da $0\leq l_i/p\leq q_i/p=\alpha_i$.
	
	Den allgemeinen Fall führen wir auf den ersten zurück, indem wir $(\alpha_1,\dots,\alpha_k)\in\R^k$ durch Brüche mit Primzahlen im Nenner annähern.
	Dazu definieren wir die Folge 
	$$
	(v^j)_{j\in\N}:=(q^j_1/p^j,\dots,q^j_k/p^j)_{j\in\N}$$
	mit $q^j_i\in\N$, $p^j$ Primzahl und $q^j_i/p^j\in[\alpha_i, \alpha_i+j^{-1}]$.
	Damit ist $\lim_{j\rightarrow\infty}v^j=(\alpha_1,\dots,\alpha_k)$ und für $j\in\N$ können wir den ersten Fall auf $u$ und $v^j$ anwenden und erhalten damit $\beta^j\in\bigtimes_{i=1}^k[0,v^j_i]$ mit $\beta^j\neq0$ sowie $\sum_{i=1}^k\beta^j_i u^i \in\Z^d$.
	Da die Folge $(\beta^j)_{j\in\N}$ in der kompakten Menge $\bigtimes_{i=1}^k[0,\alpha_i+1]$ liegt, existiert nach Satz von Bolzano-Weierstraß eine konvergente Teilfolge mit $\lim_{j\to\infty} \beta^{\sigma(j)}=:\beta$.
	Da $\beta^{\sigma(j)}_i\in[0,v^{\sigma(j)}_i]$ und $\lim_{j\to\infty}v^{\sigma(j)}_i=\alpha_i$ ist $\beta_i\in[0,\alpha_i]$.
	Da es nur endlich viele $z\in\Z^d$ der Form $z=\sum_{i=1}^k\gamma_i u^i$ mit $\gamma_i\in[0,\alpha_i+1]$ gibt, existiert ein Punkt $z\in\Z^d$, für den $\sum_{i=1}^k \beta^{\sigma(j)}_i u^i=z$ für unendlich viele $j$ gilt.
	Also gibt es wegen der Konvergenz von $(\beta^{\sigma(j)})_{j\in\N}$ ein $n$, sodass $\sum_{i=1}^k\beta^{\sigma(j)}_i u^i=z$ für $j\geq n$ und damit ist $\sum_{i=1}^k\beta_i u^i=z$.
	
	Ist $\beta\neq0$, so erfüllt es die Behauptung.
	Andernfalls ist $z=0$.
	Setze $\varepsilon>0$, sodass $\varepsilon\beta^n_i\in[0,\alpha_i]$ für $i\in\firstNumbers{k}$.
	Nach Wahl von $\beta^n$ ist dann $\varepsilon\beta^n\neq0$ und $\sum_{i=1}^k\varepsilon\beta^n_iu^i=\varepsilon z=\zero\in\Z^d$.
\end{proof}

\begin{lemma}\label{lem:maxgamma}
	Seien $u^i\in\Z^d$, $\lambda_i\geq0$ für $i\in\firstNumbers{k}$ gegeben.
	Dann existiert $\gamma\in\bigtimes_{i=1}^k [0,\lambda_i]$ mit $\sum_{i=1}^k \gamma_i u^i\in\Z^d$ und  $\sum_{i=1}^k(\lambda_i-\gamma_i)<d$.
\end{lemma}
\begin{proof}
	Die beschränkte Menge $G:=\{\gamma \in \bigtimes_{i=1}^k [0,\lambda_i] \mid \sum_{i=1}^k \gamma_i u^i\in\Z^d \}$ ist abgeschlossen: Sei $(\gamma^n)_{n\in\N}$ eine in $\R^k$ konvergente Folge mit $\gamma^n\in G$ für $n\in\N$ und $\lim_{n\to\infty}\gamma^n=:\tilde{\gamma}$.
	Da es nur endlich viele $z\in\Z^d$ gibt mit $z=\sum_{i=1}^k\gamma_iu^i$ für $\gamma_i\in[0,\lambda_i]$, existiert ein $z\in\Z^d$, sodass $z=\sum_{i=1}^k\gamma^n_iu^i$ für unendlich viele $\gamma^n$ gilt und damit auch für $\tilde{\gamma}\in G$.
	Also ist $G$ abgeschlossen und mit dem Satz von Heine-Borel ist $G\subseteq\R^k$ auch kompakt.
	Nach dem Satz von Weierstraß nimmt die Menge $\{ \sum_{i=1}^k\gamma_i \mid \gamma\in G \}$ also bei einem $\gamma\in G$ ihr Maximum an.
	
	Angenommen, es gelte $\sum_{i=1}^k(\lambda_i - \gamma_i) \geq d$.
	Wenden wir Lemma~\ref{lem:olson} an auf $\alpha_i:=\lambda_i-\gamma_i\geq0$ und $u^i$ für $i\in\firstNumbers{k}$, erhalten wir $\beta\in\bigtimes_{i=1}^k[0,\lambda_i-\gamma_i]$ mit $\beta\neq0$ und $\sum_{i=1}^k \beta_i u^i\in\Z^d$.
	Für $\gamma':=\gamma+\beta\leq\lambda$ gilt nun 
	$
	\sum_{i=1}^k\gamma'_iu^i = \sum_{i=1}^k\gamma_iu^i + \sum_{i=1}^k \beta_iu^i\in\Z^d
	$
	und damit ist $\gamma'\in G$.
	Wegen $\beta\neq0$ ist $\sum_{i=1}^k\gamma'_i > \sum_{i=1}^k\gamma_i$, was im Widerspruch zur Maximalität von $\gamma$ steht.
\end{proof}

\subsection{Abschätzung optimaler Lösungen gemischt-ganzzahliger Programme}
Dieser Abschnitt verfolgt das Ziel eine Abschätzung optimaler Lösungen zweier gemischt-ganzzahliger Programme	($I$-\MIPR) und ($J$-\MIPR) in Abhängigkeit von $\Delta$ und $\betrag{I\cup J}$ zu finden.

Zunächst betrachten wir noch folgendes Hilfslemma über die Darstellung eines Kegels:
\begin{lemma}\label{lem:cone}
	Sei $A\in\Z^{m\times n}$, dessen Zeilen in zwei Untermatrizen $A_1$ und $A_2$ aufgeteilt sind.
	Die Menge $C:=\{ x\in\R^n \mid A_1x\leq 0, A_2x\geq0 \}$ ist ein konvexer Kegel und besitzt die Darstellung \[ C=\{\lambda_1 v^1+\dots+\lambda_kv^k \mid \forall i\in\firstNumbers{k} \colon\lambda_i\geq0 \}\] mit $v^i\in\Z^n$ und $\norm{v^i}\leq\Delta$ für $i\in\firstNumbers{k}$.
\end{lemma}
\begin{proof}
	Seien $x\in C$ und $\alpha>0$ gegeben.
	Dann sind $A_1 \alpha x=\alpha A_1 x\leq 0$ und $A_2\alpha x\geq0$, also liegt $\alpha x$ in $C$.
	Also ist $C$ ein Kegel, der aufgrund der Darstellung als Polyeder insbesondere konvex ist.
	
	\todo{Darstellung: \glqq standard arguments involving Cramer's rule\grqq.}
\end{proof}

Nun formulieren wir unser Theorem:

\begin{theorem}\label{thm:theo2}
	Seien $I,J\subseteq\firstNumbers{n}$, sodass ($J$-\MIPR) eine optimale Lösung $\tilde{y}$ hat.
	Für jede optimale Lösung $x^*$ von ($I$-\MIPR) existiert eine optimale Lösung $y^*$ von ($J$-\MIPR) mit $\norm{x^*-y^*}\leq\betrag{I\cup J}\cdot\Delta$.
\end{theorem}
\begin{proof}
	Zunächst tauschen wir alle ganzzahligen Variablen an die ersten Stellen und können $I\cup J=\firstNumbers{d}$ mit $d\in\firstNumbers{n}$ annehmen.
	Es sei eine optimale Lösung $x^*$ von ($I$-\MIPR) gegeben.
	Setze $z:=\tilde{y}-x^*$ sowie $C:=\{ x\in\R^n \mid A_1 x \leq 0, A_2x\geq0 \}$, wobei die Zeilen von $A$ in $A_1$ und $A_2$ so aufgeteilt werden, dass $A_1y<0$ und $A_2y\geq0$ erfüllt werden.
	Da $y$ in $C$ liegt, erhalten wir nach Lemma~\ref{lem:cone} die Darstellung 
	$$y = \lambda_1v^1 + \dots+\lambda_kv^k$$
	mit $\lambda_i\geq0$, $\norm{v^i}\leq \Delta$ und $v^i\in C\cap\Z^n$ für alle $i\in\firstNumbers{k}$.
	Man setze $u^i$ als die Projektion des Vektors $v^i$ auf dessen erste $d$ Komponenten.
	Nach Lemma~\ref{lem:maxgamma} existiert ein $\gamma\in\bigtimes_{i=1}^k[0,\lambda_i]$ mit  $\sum_{i=1}^k\gamma_iu^i\in\Z^d$ und $\sum_{i=1}^k(\lambda_i -\gamma_i)<d$.
	
	Wir definieren nun unseren Kandidaten $$y^*:=\tilde{y}-\sum_{i=1}^k\gamma_iv^i=x^*+\sum_{i=1}^k(\lambda_i-\gamma_i)v^i$$
	sowie $$\tilde{x}:=x^*+\sum_{i=1}^k\gamma_iv^i=\tilde{y}-\sum_{i=1}^k(\lambda_i-\gamma_i)v^i.$$
	Wir zeigen zunächst, dass $y^*$ für ($J$-\MIPR) und $\tilde{x}$ für ($I$-\MIPR) zulässig sind.
	Dabei sind für $j\in J$ bzw. $i\in I$ die Koordinaten $y^*_j$ bzw. $\tilde{x}_i\in\Z$, da $\tilde{y}_j$ bzw. $x^*_i\in\Z$ und $i,j\in\firstNumbers{d}$ sowie $\sum_{l=1}^k\gamma_lv^l\in\Z^d\times\R^{n-d}$.
	$Ay^*\leq b$ und $A\tilde{x}\leq b$ folgen nun mit $v^i\in C$ für $i\in\firstNumbers{k}$ und 
	$$
	\begin{array}{lllllll}
	A_1 y^*&=&A_1x^*+\sum_{i=1}^k(\lambda_i-\gamma_i)A_1v^i &\leq& A_1 x^* &\leq&b_1\\
	A_2 y^*&=&A_2\tilde{y}-\sum_{i=1}^k\gamma_iA_2v^i &\leq& A_2\tilde{y} &\leq& b_2\\
	A_1\tilde{x} &=&A_1x^*+\sum_{i=1}^k \gamma_iA_1v^i &\leq& A_1x^* &\leq&b_1\\
	A_2\tilde{x} &=&A_2\tilde{y}-\sum_{i=1}^k(\lambda_i-\gamma_i)A_2v^i&\leq& A_2\tilde{y}&\leq& b_2.
	\end{array}
	$$
	
	Da $x^*$ optimal für ($I$-\MIPR), gilt $c\transpose x^*\geq c\transpose \tilde{x} = c\transpose x^* + c\transpose (\sum_{i=1}^k\gamma_iv^i)$ und  $c\transpose(\sum_{i=1}^k\gamma_iv^i)\leq0$.
	Damit erhalten wir mit der Optimalität von $\tilde{y}$ für ($J$-\MIPR) und
	$$c\transpose y^*=c\transpose\tilde{y}-c\transpose(\sum_{i=1}^k\gamma_iv^i)\geq c\transpose\tilde{y}$$
	auch die Optimalität von $y^*$ für ($J$-\MIPR).
	Außerdem gelten die folgenden Abschätzungen:
	$$\norm{x^*-y^*}=\norm{\sum_{i=1}^k(\lambda_i-\gamma_i)v^i}\leq \sum_{i=1}^k(\lambda_i-\gamma_i)\norm{v^i}\leq \sum_{i=1}^k(\lambda_i-\gamma_i)\Delta\leq d\Delta.
	$$
\end{proof}
\begin{remark}
	In der letzten Zeile des Beweises kann man erkennen, dass für $A\neq 0$ sogar die strikte Abschätzung $\norm{x^*-y^*}<\betrag{I\cup J}\cdot\Delta$ gilt, da dann $\Delta\neq 0$ ist.
\end{remark}

\newcommand{\one}{\mathbbm{1}}
\newcommand{\eq}[1]{{\operatorname{eq}(#1)}}
\newcommand{\co}[1]{\operatorname{co}(#1)}

\section{Lineare Abhängigkeit von $\Delta$}\label{sec:linear}

Bisher haben wir Abschätzungen mit $b\in\R^m$ betrachtet.
Schrijver hat in~\cite[Kapitel~17.2]{Schrijver1986} ein Beispiel angeführt, das $n\Delta$ als beste Abschätzung von optimalen Lösungen von ($\emptyset$-\MIPR) und ($\firstNumbers{n}$-\MIPR) besitzt.
Für $b\in\Z^m$ erkennen wir mit folgendem Beispiel, dass der Abstand zumindest linear von $\Delta$ abhängt:
\begin{example}
	Für $\delta\in\N$ sei
	$$A:=
	\begin{pmatrix}
	-\delta & 0  \\
	\delta  & -1
	\end{pmatrix},\quad
	b:=\begin{pmatrix} -1 \\ 0 \end{pmatrix},\quad
	c\transpose:=\begin{pmatrix}0 & -1\end{pmatrix}.
	$$
	Es wird also die zweite Komponente minimiert unter der Nebenbedingung $Ax\leq b$, also $\delta x_1-x_2\leq0\Leftrightarrow\delta x_1\leq x_2$ und $\delta x_1\geq 1$.
	Es gilt $\Delta=\delta$ und die optimale Lösung von ($\emptyset$-\MIPI) und (\{2\}-\MIPI) ist $x^*=(1/\delta,1)$.
	Die optimale Lösung von $(\{1\}-\MIPI)$ und ($\{1, 2\}$-\MIPI) ist jedoch $y^*=(1,\delta)$.
	Entsprechend ist der Abstand $\norm{x^*-y^*}=\delta-1=\Omega(\Delta)$
\end{example}

\begin{notation}
	Seien $A\in\R^{m\times n}$, $b\in\R^m$ und $x^*\in\R^n$ gegeben.
	Im Kontext des Polyeders $\{ x\in\R^m \mid Ax\leq b \}$ bezeichnen
	$A_\eq{x^*}$ und $b_\eq{x^*}$ diejenigen Zeilen aus $A$ und $b$, bei denen für die zugehörigen Ungleichungen mit $x^*$ sogar Gleichheit gilt.
	
	Für eine Menge $M\subseteq \R^n$ bezeichne $\co{M}$ ihre konvexe Hülle.
\end{notation}

Wir benutzen nun das folgende Lemma aus~\cite[Theorem 2 und Beweis]{VESELOV2009220}, um für einige Situationen die Vermutung~\ref{con:delta} zu bestätigen:
\begin{lemma}\label{lem:veselov}
	Sei $A\in\Z^{m\times n},b\in\Z^m,c\in\R^n$ mit $\rang(A)=n$, sodass der Absolutwert jeder Determinante einer $n\times n$ Teilmatrix kleinergleich 2 ist.
	Seien $z$ eine Ecke von $P:=\{x\in\R^n\mid Ax\leq b \}$ und $Q$ die konvexe Hülle aller ganzen Zahlen, die alle Ungleichungen aus $Ax\leq b$ erfüllen, die an $z$ straff sind.
	Dann gelten:
	\begin{enumerate}[(a)]
		\item Jede Ecke von $Q$ liegt auf einer Kante von $P$, die $z$ enthält.
		\item Jede Kante von $P$, die $z$ und einen ganzzahligen Punkt enthält, enthält auch einen ganzzahligen Punkt $y^*$ mit $\norm{z -y^*}\leq 1$.
	\end{enumerate}
\end{lemma}
\begin{lemma}\label{lem:bounded}
	Seien $A\in\R^{m\times n}$, $b\in\R^m$ und $U\in\N$ gegeben.
	Mit $$
		\tilde{A}:=
		\begin{pmatrix} A \\ -\one_n \\ \one_n \end{pmatrix},\quad
		\tilde{b}:=\begin{pmatrix} b \\ U \\ U \end{pmatrix}
	$$
	% ist $\{ x\in\R^n \mid Ax\leq b \}\cap\{x\in\R^n \mid \forall i\in\firstNumbers{m}: -U \leq x_i \leq U \} = \{ x\in\R^n \mid \tilde{A}x\leq \tilde{b}  \}$ und es 
	gilt $\Delta(A)=\Delta(\tilde{A})$.
\end{lemma}
\begin{proof}
	Da jede Untermatrix von $A$ auch eine Untermatrix von $\tilde{A}$ ist, folgt die Ungleichung $\Delta(\tilde{A})\leq\Delta(A)$.
	Sei nun $M$ eine quadratische Untermatrix von $\tilde{A}$.
	Die ersten Zeilen $M_1$ von $M$ sind dabei eine Untermatrix von $A$, die letzten Zeilen $M_2$ eine Untermatrix von $(-\one, \one)\transpose$.
	Existiert eine Spalte, in der in $M_2$ sowohl eine $-1$ als auch eine $1$ vorkommt, so sind die jeweiligen Zeilen linear abhängig und $\det(M)=0$.
	Sonst lässt sich $M$ mit elementaren Zeilen- und Spaltentransformationen in die Form $$\tilde{M}:= \begin{pmatrix}
		B & 0 \\
		0 & \one
	\end{pmatrix}$$ bringen mit $B$ Untermatrix von $A$ und $\betrag{\det(M)}=\betrag{\det(\tilde{M})}=\betrag{\det(B)}\leq\Delta(A)$.
\end{proof}

\begin{lemma}\label{lem:unimodular}
	Seien $A\in\R^{m\times n},b\in\Z^m$ gegeben mit $A$ ist total unimodular, d.h. $\Delta(A)\leq 1$. Dann ist jede Ecke von $\{x\in\R^n \mid Ax\leq b \}$ ganzzahlig.
\end{lemma}
\begin{proof}
	Sei $x$ eine Ecke. Dann ist $\rang(\tilde{A})=n$, wobei $\tilde{A} := A_\eq{x}$ und $\tilde{b}:=b_\eq{x}$ aus den Zeilen von $A$ und $b$ besteht, die straffe Bedingungen an $x$ darstellen, d.h. $\tilde{A} x = \tilde{b}$.
	Nun kann $x$ mit der Cramerschen Regel eindeutig gelöst werden mit $x_i=\det(\tilde{A}_i)/\det(\tilde{A})$, wobei $\tilde{A}_i$ aus $\tilde{A}$ durch Ersetzen der $i$-ten Spalte mit $\tilde{b}$ entsteht.
	Da $\det(\tilde{A})\in\{-1,1\}$ wegen $\Delta(A)\leq 1$, ist $x_i$ ganzzahlig.
\end{proof}
\begin{lemma}\label{lem:q-upper-bound}
	Seien $A\in\R^{m\times n}$, $b\in\R^m$ und $c\in\R^n$ gegeben.
	Mit $P:=\{ x\in\R^n \mid Ax\leq b \}$ gilt für jedes $x^*\in P$, das $x\mapsto c\transpose x$ maximiert:
	Über $Q:=\co{\{x\in\Z^n \mid A_\eq{x^*}x \leq b_\eq{x^*} \}}$ ist  $x\mapsto c\transpose x$ mit $c\transpose x^*$ nach oben beschränkt.
\end{lemma}
\begin{proof}
	Für beliebiges $q\in Q\setminus\{x^*\}$ gilt:
	Es existiert $\lambda\in[0,1)$ mit $\tilde{q}:=\lambda q+(1-\lambda)x^*\in P$, da wir uns für jede Ungleichung, die $q$ noch nicht erfüllt, solange an $x^*$ annähern können bis sie erfüllt ist und diese Ungleichung für $x^*$ keine Gleichheit erfüllt.
	Ist $\lambda=0$, so ist $q\in P$ und $c\transpose q\leq c\transpose x^*$. Mit $c\transpose \tilde{q} \leq c\transpose x^*$ gilt sonst $c\transpose q=(c\transpose \tilde{q} - c\transpose(1-\lambda)x^*)/\lambda \leq c\transpose x^*$.
	Also ist $x\mapsto c\transpose x$ in $Q$ mit $c\transpose x^*$ nach oben beschränkt.
\end{proof}
Damit können wir nun folgendes Theorem formulieren:
\begin{theorem}
		Seien $\Delta\leq 2$ und $I,J\in\{\emptyset,\firstNumbers{n}\}$, sodass eine optimale Lösung von ($J$-\MIPI) existiert.
		Für jede optimale Lösung $x^*$ von ($I$-\MIPI) existiert eine optimale Lösung $y^*$ von ($J$-\MIPI) mit $\norm{x^*-y^*}\leq \Delta$.
\end{theorem}
\begin{proof}
	Sei $x^*$ eine optimale Lösung von ($I$-\MIPI) gegeben.
	Für $\Delta=0$ ist $A$ unimodular und nach Lemma~\ref{lem:unimodular} ist $x^*\in\Z^n$ und optimal für ($\emptyset$-\MIPI).
	Wir betrachten also $\Delta\in\{1,2\}$.
	
	Es existiert ein $U\in\N$, sodass die beschränkte Menge  $$P:=\{x\in\R^n\mid Ax\leq b\} \cap \{x\in\R^n \mid \forall i\in\firstNumbers{m}: -U \leq x_i \leq U \}$$
	$x^*$ und eine optimale Lösung von ($J$-\MIPI) enthält.
	Setzt man $\tilde{A},\tilde{b}$ wie in Lemma~\ref{lem:bounded} ist $\rang(\tilde{A})=n$ und $\Delta:=\Delta(A)=\Delta(\tilde{A})$ und es gilt $P=\{ x\in\R^n \mid \tilde{A}x \leq \tilde{b} \}$.
	Es genügt nun, ein $y^*$ in $P$ zu finden, das für ($J$-\MIPI) optimal ist und $\norm{x^* - y^*}\leq\Delta$ erfüllt.
	
	\begin{description}
		\item[Fall 1:]  $I=\emptyset,J=\firstNumbers{n}, x^*$ ist Ecke von $P$.
		
		Ist $\Delta= 1$, so ist $x^*$ nach Lemma~\ref{lem:unimodular} ganzzahlig und $y^* := x^*$ ist die gewünschte Lösung mit $\norm{x^*-y^*}\leq \Delta - 1$.
		
		Sei also $\Delta=2$. Setze $Q:=\co{\{x\in\Z^n \mid A_\eq{x^*}x \leq b_\eq{x^*} \}}$.
		Da $Q$ nicht-leer ist und $x\mapsto c\transpose x$ nach Lemma~\ref{lem:q-upper-bound} mit $c\transpose x^*$ über $Q$ nach oben beschränkt ist, existiert also eine Ecke $z\in\Z^n$ von $Q$, die $x\mapsto c\transpose x$ maximiert.
		Nach Lemma~\ref{lem:veselov}~(a) liegt $z$ auf einer Kante $E$ von $P$, die $x^*$ enthält.
		
		Sei $y^*\in\Z^n\cap E$ mit $\norm{x^* - y^*} := \min\{ \norm{x^*-y} \mid y\in\Z^n\cap E  \}$.
		Es existiert ein $\lambda\in [0,1]$ mit $y^* = \lambda z + (1-\lambda)x^*$ und mit $c\transpose x^* \leq c\transpose z$ gilt $c\transpose y^* \geq c\transpose z$.
		Aus der Optimalität von $z$ folgt die Optimalität von $y^*$ für ($\firstNumbers{n}$-\MIPI) und mit Lemma~\ref{lem:veselov}~(b) folgt aus der Wahl von $y^*$ die Abschätzung $\norm{x^*-y^*}\leq 1=\Delta-1$.

		\item[Fall 2:] $I=\emptyset$, $J=\firstNumbers{n}$.
		
		Die Menge aller optimalen Lösungen von ($\emptyset$-\MIPI) bildet eine Seitenfläche von $P$, nämlich $F:=\{x\in P\mid -c\transpose x \leq cx^* \}$.
		Sei $\tilde{z}$ eine optimale Lösung von ($\firstNumbers{n}$-\MIPI) und $\tilde{x}$ eine Ecke des Polytops $B:=\{ x\in F \mid \forall i \in \firstNumbers{n}: \Floor{x^*_i} \leq x_i \leq \Ceil{x^*_i} \}$.
		Dann folgt bereits $\norm{\tilde{x}-x^*}\leq 1$.
		
		Definiere die Polyeder
		$$
		\begin{array}{l}
		P_1:= \{ x\in\R^n \mid \forall i\in\firstNumbers{n}: (\tilde{y}_i\leq \Floor{x^*_i} = \tilde{x}_i \Rightarrow x_i \leq \Floor{x^*_i})\},\\
		P_2:= \{ x\in\R^n \mid \forall i\in\firstNumbers{n}: (\tilde{y}_i\geq \Floor{x^*_i} = \tilde{x}_i \Rightarrow x_i \geq \Floor{x^*_i})\}, \\
		P_3:= \{ x\in\R^n \mid \forall i\in\firstNumbers{n}: (\tilde{y}_i\leq \Ceil{x^*_i} = \tilde{x}_i \Rightarrow x_i \leq \Ceil{x^*_i})\},\\
		P_4:= \{ x\in\R^n \mid \forall i\in\firstNumbers{n}: (\tilde{y}_i\geq \Ceil{x^*_i} = \tilde{x}_i \Rightarrow x_i \geq \Ceil{x^*_i})\}
		\end{array}
		$$
		und das Polytop $\tilde{P}:=P\cap P_1 \cap P_2 \cap P_3 \cap P_4$.
		$\tilde{P}$ ist nicht-leer, da $\tilde{x}, \tilde{y}\in\tilde{P}$, und beschränkt, da $\tilde{P}\subseteq P$.
		Da jede an $\tilde{x}$ straffe Ungleichung von $B$ auch in $\tilde{P}$ vorkommt, ist $\tilde{x}$ auch eine optimale Ecke von $\tilde{P}$.
		
		$\tilde{P}$ kann wieder durch eine ganzzahlige Matrix $\tilde{A}$ beschrieben werden, die weiterhin $\rang(\tilde{A})=n$ und $\Delta(\tilde{A})=\Delta(A)$ \todo{Beweis?} erfüllt.
		Nun können wir Fall 1 anwenden und erhalten eine optimale Lösung $y^*$ in $\tilde{P}\cap\Z^n$, also auch eine optimale Lösung von ($\firstNumbers{n}$-\MIPI), mit $\norm{\tilde{x} -y^*}\leq \Delta -1$.
		Mit Dreiecksungleichung folgt $\norm{x^* - y^*}\leq \Delta$.
		
		\item[Fall 3:] $I=\firstNumbers{n},J=\emptyset,x^*$ ist Ecke von $R:=\co{\{ x\in\Z^n \mid Ax\leq b \}}$.
		
		Für $\Delta=1$ ist $x^*$ nach Lemma~\ref{lem:unimodular} auch eine optimale Lösung von ($\emptyset$-\MIPI).
		Wir betrachten den Fall $\Delta=2$.
		Nach Definition einer Ecke, existiert ein Vektor $d\in\R^n$ mit $\{ x\in\R^n \mid \forall \tilde{x}\in R: d\transpose x \geq d\transpose\tilde{x} \}=\{x^*\}$.
		Demnach ist $d\transpose x < d\transpose x^*$ für alle $x\in R\setminus\{x^*\}$.
		Für die Seitenfläche $F\subseteq P$, die alle optimalen Lösungen von ($\emptyset$-\MIPI) enthält, kann man nun ein $\lambda\geq0$ finden, das groß genug ist, dass eine Ecke $z\in F$ existiert, die $x\mapsto (\lambda c+d)\transpose x$ über $P$ maximiert. \todo{Beweis Existenz $\lambda$?}
		
		Mit $\tilde{c}:=\lambda c+d$ gilt für jeden Punkt $x\in R\setminus\{x^*\}$:
		$$\tilde{c}\transpose x < \lambda c\transpose x + d\transpose x^*\leq \lambda c\transpose x^* + d\transpose x^* = \tilde{c}\transpose x^*.$$
		Setze $Q:=\co{\{x\in\Z^n \mid A_\eq{z}x \leq b_\eq{z} \}}$.
		Nach Lemma~\ref{lem:q-upper-bound} ist $x\mapsto \tilde{c}\transpose x$ mit $\tilde{c}\transpose z$ in $Q$ nach oben beschränkt und, da $Q$ nicht-leer ist, gibt es eine Ecke $v\in\Z^n$ von $Q$, die $x\mapsto\tilde{c}\transpose x$ über $Q$ maximiert.
		Nach Lemma~\ref{lem:veselov}~(a) liegt $v$ auf einer Kante $E$ von $P$, die $z$ enthält.
		Insbesondere ist also $Av\leq b$ und damit $v\in R$.
		Da auch $R \subseteq Q$, ist $v$ Maximierer von $x\mapsto \tilde{c}\transpose x$ über $R$ und damit ist $v=x^*$.
		
		Für den Fall $c\transpose x^*=c\transpose z$ ist $x^*$ bereits optimal für ($\emptyset$-\MIPI).
		Sonst ist $x^* \neq z$ und, da $v=x^*$, liegt $x^*$ auf der Kante $E$ von $P$, die $z$ enthält.
		Auf der offenen Strecke zwischen $x^*$ und $z$ liegen keine ganzzahligen Punkte, weil $x\mapsto \tilde{c}\transpose x$ von $x^*$ über $R$ und von $z$ über $P\supseteq R$ maximiert und beide auf einer Kante von $P$ liegen.
		Also gilt nach Lemma~\ref{lem:veselov}~(b), dass $\norm{x^*-z}\leq 1 = \Delta -1 $
		
		\item[Fall 4:] $I=\firstNumbers{n},J=\emptyset.$
		
		Da $P$ beschränkt, ist auch $R:=\co{\{x\in\Z^n \mid Ax\leq b\}}$ beschränkt und, da $x^*\in R$, existiert eine Konvexkombination $x^* = \sum_{i=1}^t \lambda_i v^i$ mit $v^1,\dots,v^t$ Ecken von $R$ und $\lambda_1,\dots,\lambda_t>0$ und $\sum_{i=1}^t \lambda_i=1$.
		Angenommen für ein $k\in\firstNumbers{t}$ ist $v^k$ nicht optimal für ($\firstNumbers{n}$-\MIPI).
		Dann ist $c\transpose x^*=\sum_{i=1}^t \lambda_i c\transpose v^i \leq (1-\lambda_k)c\transpose x^* + \lambda_k c\transpose v^k < c\transpose x^*$.
		Also ist $v^i$ optimal für ($\firstNumbers{n}$-\MIPI) und nach Fall 3 existiert $z^i\in\R^n$ optimal für ($\emptyset$-\MIPI) mit $\norm{v^i-z^i}\leq \Delta$ für alle $i\in\firstNumbers{t}$.
		Die Konvexkombination $y^*:=\sum_{i=1}^t \lambda_i z^i$ ist ebenfalls optimal für ($\emptyset$-\MIPI) und
		$$\norm{x^*-y^*}=\norm{\sum_{i=1}^t \lambda_i (v^i-z^i)}\leq \sum_{i=1}^t \lambda_i \norm{v^i-z^i}=\Delta.$$
	\end{description}
\end{proof}

\clearpage          % neue Seite für Literaturverzeichnis

%%%%%%%%%%%%%%%%%%%%%%%%%%%%%%%%%%%%%%%%%%%%%%%%%%%%%%%%%%%%%%%%%%%%%%%%%%%%%%%%
% Literaturverzeichnis
\nocite*  % Nicht zitierte Quellen werden auch ins Literaturverzeichnis aufgenommen
\thispagestyle{empty}
\bibliography{Literatur/seminararbeit}  % Literaturverzeichnis liegt in der Datei seminararbeit

%%%%%%%%%%%%%%%%%%%%%%%%%%%%%%%%%%%%%%%%%%%%%%%%%%%%%%%%%%%%%%%%%%%%%%%%%%%%%%%%
%%%%%%%%%%%%%%%%%%%%%%%%%%%%%%%%%%%%%%%%%%%%%%%%%%%%%%%%%%%%%%%%%%%%%%%%%%%%%%%%
% Ende des Dokuments
\end{document}		
