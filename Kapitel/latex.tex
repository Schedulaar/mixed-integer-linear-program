\section{Erläuterungen zu \LaTeX} 

Einige Beispiele für die mathematischen Umgebungen und deren Verwendung:

\begin{satz}[Betrag ist nicht negativ]\label{Satz:Betrag}\hspace{0em}\\
 $ \betrag{x} $ beziehungsweise $ \norm{x} $ ist für $ x\in \mathbb{R} $ oder $ x \in \mathbb{R}^{n} $ immer nicht negativ.
\end{satz}

Der Satz wird mit \verb+\label{Satz:Betrag}+ gekennzeichnet und anschließend im Beweis mit \verb+\autoref{Satz:Betrag}+ verknüpft, damit ergibt sich statt der Darstellung \ref{Satz:Betrag} mit dem einfachen Befehl \verb+\ref{Satz:Betrag}+, eine um die Bezeichnung Satz erweiterte Darstellung.

\begin{proof}[Beweis zu \autoref{Satz:Betrag}]
 Nun ja trivial.
\end{proof}

Weitere Beispiele für die unterschiedlichen Spielarten von Theoremen:

\begin{definition}[Seminar]\label{Def:Seminar}
 Ein Seminar besteht aus einer schriftlichen Ausarbeitung sowie einem mündlichen Vortrag über ein bestimmtes vom Betreuer vorgegebenes Thema.
\end{definition}

\begin{bemerkung}
Das Verständnis des dem Seminars aus \autoref{Def:Seminar} zugrunde liegenden Stoffes kann zusätzlich zum eigentlichen Vortrag durch weitergehende Fragen oder durch eine Diskussion geprüft werden.
\end{bemerkung}

\begin{notation} 
Es ist stets darauf zu achten, mathematische Bezeichnungen in Mathemodus (\$ Zeichen \$) zu setzen, auch wenn diese aus nur einem Buchstaben bestehen, zum Beispiel Graph $G$, Index $k$, etc. Weiter ist es bei englischen Originaltexten oft verständlicher, auch die Notation ins Deutsche zu übertragen, z.B. anstatt von einer ungeraden Zahl $o$ zu sprechen (\textit{odd}), ist $u$ hier intuitiver.
\end{notation} 

\vspace{1em}
Zitate aus Literaturstellen werden mit \verb+\cite{Jungnickel:1999}+ gekennzeichnet; wenn die Zitatstelle noch genauer spezifiziert werden soll, dann sollte die erweiterte Version \verb+\cite[Teil4]{Borgwardt:2001}+  verwendet werden.

Das liefert dann folgende Ergebnisse:
\begin{itemize}
 \item Ein einführendes Buch zur Optimierung ist zum Beispiel \cite{Jungnickel:1999}.
 \item Eine einführende übersicht zur Spieltheorie findet sich in \cite[Teil 4]{Borgwardt:2001}.
\end{itemize}

Werden mathematische Formeln in eine besondere mathematische Umgebung gesetzt (z.B. {\verb equation }), so wird diese nummeriert und über {\verb \label{EQ:key} } kann die Formel gekennzeichnet werden und dann über {\verb \eqref{EQ:key} } darauf Bezug genommen werden. 

\begin{equation}\label{EQ:Potenzreihe}
 \sum_{i=0}^{\infty} r^{i} = \frac{1}{1 - r} \qquad \forall r \in (-1,1)
\end{equation}

Aus \eqref{EQ:Potenzreihe} folgt, dass die Potenzreihe von $ \frac{1}{2} $ den Grenzwert $ 2 $ hat.

Komfortablere mehrzeilige mathematische Umgebungen als \textit{eqnarray} finden sich im Paket \textit{amsmath}, eine kurze Beschreibung dieses Paketes findet sich zum Beispiel unter: 
\begin{center}
\href{ftp://dante.ctan.org/tex-archive/macros/latex/required/amslatex/math/amsldoc.pdf}{ftp://dante.ctan.org/tex-archive/macros/latex/required/amslatex/math/amsldoc.pdf}                               
\end{center}

Weiterhin kann man auf den Webseiten des Lehrstuhls \lehrstuhl\, unter Nützliches $\rightarrow$ \LaTeX \ eine allgemeine Sammlung von interessanten Links zu \LaTeX \ finden.
