\section{Gemischt-ganzzahlige lineare Programme}\label{introduction}

Diese Arbeit behandelt die Abschätzung von Abständen optimaler Lösungen
gemischt-ganzzahliger Programme.
Ein gemischt-ganzzahliges Programm ist ein lineares Programm, bei dem die
Matrix, die rechte Seite $b$ sowie einige Variablen ganzzahlig und die
restlichen Variablen reell sind.
Speziell erzeugen $A\in\Z^{n\times m}$, $b\in\Z^m$, $c\in\R^n$ und $I\subseteq\firstNumbers{n}:=\{1,\dots,n\}$ das gemischt-ganzzahlige Programm (mixed integer program, MIP)
\begin{equation}\tag{$I$-\MIPI}
\begin{array}{lc}
	&\max c\transpose x \\
	\subjectTo &Ax\leq b\\
	&\forall i\in I: x_i\in\Z
\end{array}.
\end{equation}

Die Notation ($I$-\MIPR) wird verwendet, falls zusätzlich $b\in\R^m$ zugelassen wird.
Ist also beispielsweise $I=\emptyset$, erhält man ein Problem in Standardform; $I=\firstNumbers{n}$ erzeugt ein rein ganzzahliges lineares Programm.

Gesucht ist nun für $I,J\subseteq\firstNumbers{n}$ eine möglichst kleine Schranke, die den Abstand zwischen jeder optimalen Lösung $x^*$ von ($I$-\MIPI) und einer zu $x^*$ nähesten optimalen Lösung $y^*$ von ($J$-\MIPI) beschränkt.
Diese Schranke soll außerdem nur von $A$, $I$ und $J$ abhängen und der Abstand mittels Maximumsnorm ermittelt werden.
Desweiteren wird immer vorausgesetzt, dass ($J$-\MIPI) eine optimale Lösung besitzt.
Definiert man
$$\Delta:=\Delta(A):=\max\{\betrag{ \det(Q)} \mid Q \text{ quadratische Untermatrix von } A \},$$
so erhält man die folgende in \cite{Paat2018} formulierte Vermutung.

\begin{conjecture}\label{con:delta}
	Es gibt eine Funktion $f: \N\rightarrow\R$, sodass für alle $I,J\subseteq\firstNumbers{n}$, unter denen ($J$-\MIPI) eine optimale Lösung besitzt, gilt:
	Für jede Lösung $x^*$ von ($I$-\MIPI) existiert eine optimale 
	Lösung $y^*$ von ($J$-\MIPI) mit $\norm{x^* - y^*}\leq f(\Delta)$.
\end{conjecture}

\begin{remark}\label{rem:feasibility}
	Ein Korollar aus dem Theorem von Meyer in~\cite[Korollar 5.2]{Meyer1974} besagt, dass ein zulässiges gemischt-ganzzahliges Problem genau dann eine optimale Lösung besitzt, wenn das entsprechende unrestringierte Problem, bei dem keine Entscheidungsvariable als ganzzahlig gefordert ist, eine optimale Lösung besitzt.
	
	In der Vermutung~\ref{con:delta} kann also die Voraussetzung der Existenz einer optimalen Lösung von ($J$-\MIPI) ersetzt werden, falls im Gegenzug  nur Zulässigkeit von ($J$-\MIPI) verlangt wird.
	Dies gilt unabhängig davon, ob $b$ in $\Z^m$ liegt.
	
	Um dieses Argument nicht unnötigerweise oft zu wiederholen, wurde diese Ersetzung hier und in folgenden Lemmata und Theoremen nicht durchgeführt.
\end{remark}
\begin{proof}
	Sei eine optimale Lösung $x^*$ von ($I$-\MIPI) gegeben.
	Nach dem genannten Korollar existiert also eine Optimallösung für ($\emptyset$-\MIPI).
	Da ($J$-\MIPI) zulässig ist, kann das Korollar erneut angewandt werden -- nun in die andere Richtung --, wodurch man die Existenz einer optimalen Lösung von ($J$-\MIPI) erhält.
\end{proof}

In dieser Arbeit wird eine etwas schwächere Aussage in Theorem~\ref{thm:theo2} gezeigt, bei der der Abstand zusätzlich noch von $\betrag{I\cup J}$ abhängt.
Des Weiteren werden in Abschnitt~\ref{sec:linear} Fälle diskutiert, in denen sogar die verstärkte Vermutung der Linearität von $f$ bewiesen werden kann.
