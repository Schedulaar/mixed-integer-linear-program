\section{Einführung}\label{introduction}

Diese Arbeit behandelt die Abschätzung von Abständen optimaler Lösungen
gemischt-ganzzahliger Programme.
Ein gemischt-ganzzahliges Programm ist ein lineares Programm, bei dem die
Matrix, die rechte Seite $b$ sowie einige Variablen ganzzahlig und die
restlichen Variablen reell sind.
Speziell erzeugen $A\in\Z^{n\times m}$, $b\in\Z^m$, $c\in\R^n$ und $I\subseteq\firstNumbers{n}:=\{1,\dots,n\}$ das gemischt-ganzzahlige Programm
\begin{equation}\tag{$I$-\MIPI}
\begin{array}{lc}
	&\max c\transpose x \\
	\subjectTo &Ax\leq b\\
	&\forall i\in I: x_i\in\Z
\end{array}.
\end{equation}

Wir schreiben ($I$-\MIPR), falls zusätzlich $b\in\R^m$ zugelassen wird.
Ist also beispielsweise $I=\emptyset$, erhält man ein Problem in Standardform; $I=\firstNumbers{n}$ bildet ein rein ganzzahliges lineares Programm.

Gesucht ist nun für $I,J\subseteq\firstNumbers{n}$ eine möglichst kleine Schranke, die den Abstand zwischen jeder optimalen Lösung $x^*$ von ($I$-\MIPI) und einer nähesten optimalen Lösung $y^*$ von ($J$-\MIPI) beschränkt.
Diese Schranke soll außerdem nur von $A$, $I$ und $J$ abhängen und der Abstand mittels Maximumsnorm ermittelt werden.
Desweiteren wird immer vorausgesetzt, dass ($J$-\MIPI) eine optimale Lösung besitzt.
Definiert man
$$\Delta:=\Delta(A):=\max\{\betrag{ \det(Q)} \mid Q \text{ quadratische Untermatrix von } A \},$$
so erhält man die folgende in \cite{Paat2018} formulierte Vermutung.

\begin{conjecture}\label{con:delta}
	Es gibt eine Funktion $f: \N\rightarrow\R$, sodass für alle $I,J\subseteq\firstNumbers{n}$, unter denen ($J$-\MIPI) eine optimale Lösung besitzt, gilt:
	Besitzt ($I$-\MIPI) eine optimale Lösung $x^*$, so existiert eine optimale 
	Lösung $y^*$ von ($J$-\MIPI) mit $\norm{x^* - y^*}\leq f(\Delta)$.
\end{conjecture}

In dieser Arbeit werden wir eine etwas schwächere Aussage in Theorem~\ref{thm:theo2} zeigen, bei der der Abstand zusätzlich noch von $\betrag{I\cup J}$ abhängt.
Des Weiteren werden wir in Abschnitt~\ref{sec:linear} Fälle diskutieren, in denen wir sogar die verstärkte Vermutung der Linearität von $f$ beweisen können.
