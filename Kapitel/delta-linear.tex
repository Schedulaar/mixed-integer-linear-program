\newcommand{\one}{\mathbbm{1}}
\newcommand{\eq}[1]{{\operatorname{eq}(#1)}}
\newcommand{\co}[1]{\operatorname{co}(#1)}

\section{Lineare Abhängigkeit von $\Delta$}\label{sec:linear}

Bisher haben wir Abschätzungen mit $b\in\R^m$ betrachtet.
Schrijver hat in~\cite[Kapitel~17.2]{Schrijver1986} ein Beispiel angeführt, das $n\Delta$ als beste Abschätzung von optimalen Lösungen von ($\emptyset$-\MIPR) und ($\firstNumbers{n}$-\MIPR) besitzt.
Für $b\in\Z^m$ erkennen wir mit folgendem Beispiel, dass der Abstand zumindest linear von $\Delta$ abhängt:
\begin{example}
	Für $\delta\in\N$ sei
	$$A:=
	\begin{pmatrix}
	-\delta & 0  \\
	\delta  & -1
	\end{pmatrix},\quad
	b:=\begin{pmatrix} -1 \\ 0 \end{pmatrix},\quad
	c\transpose:=\begin{pmatrix}0 & -1\end{pmatrix}.
	$$
	Es wird also die zweite Komponente minimiert unter der Nebenbedingung $Ax\leq b$, also $\delta x_1-x_2\leq0\Leftrightarrow\delta x_1\leq x_2$ und $\delta x_1\geq 1$.
	Es gilt $\Delta=\delta$ und die optimale Lösung von ($\emptyset$-\MIPI) und (\{2\}-\MIPI) ist $x^*=(1/\delta,1)$.
	Die optimale Lösung von $(\{1\}-\MIPI)$ und ($\{1, 2\}$-\MIPI) ist jedoch $y^*=(1,\delta)$.
	Entsprechend ist der Abstand $\norm{x^*-y^*}=\delta-1=\Omega(\Delta)$
\end{example}

\begin{notation}
	Seien $A\in\R^{m\times n}$, $b\in\R^m$ und $x^*\in\R^n$ gegeben.
	Im Kontext des Polyeders $\{ x\in\R^m \mid Ax\leq b \}$ bezeichnen
	$A_\eq{x^*}$ und $b_\eq{x^*}$ diejenigen Zeilen aus $A$ und $b$, bei denen für die zugehörigen Ungleichungen mit $x^*$ sogar Gleichheit gilt.
	
	Für eine Menge $M\subseteq \R^n$ bezeichne $\co{M}$ ihre konvexe Hülle.
\end{notation}

Wir benutzen nun das folgende Lemma aus~\cite[Theorem 2 und Beweis]{VESELOV2009220}, um für einige Situationen die Vermutung~\ref{con:delta} zu bestätigen:
\begin{lemma}\label{lem:veselov}
	Sei $A\in\Z^{m\times n},b\in\Z^m,c\in\R^n$ mit $\rang(A)=n$, sodass der Absolutwert jeder Determinante einer $n\times n$ Teilmatrix kleinergleich 2 ist.
	Seien $z$ eine Ecke von $P:=\{x\in\R^n\mid Ax\leq b \}$ und $Q$ die konvexe Hülle aller ganzen Zahlen, die alle Ungleichungen aus $Ax\leq b$ erfüllen, die an $z$ straff sind.
	Dann gelten:
	\begin{enumerate}[(a)]
		\item Jede Ecke von $Q$ liegt auf einer Kante von $P$, die $z$ enthält.
		\item Jede Kante von $P$, die $z$ und einen ganzzahligen Punkt enthält, enthält auch einen ganzzahligen Punkt $y^*$ mit $\norm{z -y^*}\leq 1$.
	\end{enumerate}
\end{lemma}
\begin{lemma}\label{lem:bounded}
	Seien $A\in\R^{m\times n}$, $b\in\R^m$ und $U\in\N$ gegeben.
	Mit $$
		\tilde{A}:=
		\begin{pmatrix} A \\ -\one_n \\ \one_n \end{pmatrix},\quad
		\tilde{b}:=\begin{pmatrix} b \\ U \\ U \end{pmatrix}
	$$
	% ist $\{ x\in\R^n \mid Ax\leq b \}\cap\{x\in\R^n \mid \forall i\in\firstNumbers{m}: -U \leq x_i \leq U \} = \{ x\in\R^n \mid \tilde{A}x\leq \tilde{b}  \}$ und es 
	gilt $\Delta(A)=\Delta(\tilde{A})$.
\end{lemma}
\begin{proof}
	Da jede Untermatrix von $A$ auch eine Untermatrix von $\tilde{A}$ ist, folgt die Ungleichung $\Delta(\tilde{A})\leq\Delta(A)$.
	Sei nun $M$ eine quadratische Untermatrix von $\tilde{A}$.
	Die ersten Zeilen $M_1$ von $M$ sind dabei eine Untermatrix von $A$, die letzten Zeilen $M_2$ eine Untermatrix von $(-\one, \one)\transpose$.
	Existiert eine Spalte, in der in $M_2$ sowohl eine $-1$ als auch eine $1$ vorkommt, so sind die jeweiligen Zeilen linear abhängig und $\det(M)=0$.
	Sonst lässt sich $M$ mit elementaren Zeilen- und Spaltentransformationen in die Form $$\tilde{M}:= \begin{pmatrix}
		B & 0 \\
		0 & \one
	\end{pmatrix}$$ bringen mit $B$ Untermatrix von $A$ und $\betrag{\det(M)}=\betrag{\det(\tilde{M})}=\betrag{\det(B)}\leq\Delta(A)$.
\end{proof}

\begin{lemma}\label{lem:unimodular}
	Seien $A\in\R^{m\times n},b\in\Z^m$ gegeben mit $A$ ist total unimodular, d.h. $\Delta(A)\leq 1$. Dann ist jede Ecke von $\{x\in\R^n \mid Ax\leq b \}$ ganzzahlig.
\end{lemma}
\begin{proof}
	Sei $x$ eine Ecke. Dann ist $\rang(\tilde{A})=n$, wobei $\tilde{A} := A_\eq{x}$ und $\tilde{b}:=b_\eq{x}$ aus den Zeilen von $A$ und $b$ besteht, die straffe Bedingungen an $x$ darstellen, d.h. $\tilde{A} x = \tilde{b}$.
	Nun kann $x$ mit der Cramerschen Regel eindeutig gelöst werden mit $x_i=\det(\tilde{A}_i)/\det(\tilde{A})$, wobei $\tilde{A}_i$ aus $\tilde{A}$ durch Ersetzen der $i$-ten Spalte mit $\tilde{b}$ entsteht.
	Da $\det(\tilde{A})\in\{-1,1\}$ wegen $\Delta(A)\leq 1$, ist $x_i$ ganzzahlig.
\end{proof}

Damit können wir nun folgendes Theorem formulieren:
\begin{theorem}
		Seien $\Delta\leq 2$ und $I,J\in\{\emptyset,\firstNumbers{n}\}$, sodass eine optimale Lösung von ($J$-\MIPI) existiert.
		Für jede optimale Lösung $x^*$ von ($I$-\MIPI) existiert eine optimale Lösung $y^*$ von ($J$-\MIPI) mit $\norm{x^*-y^*}\leq \Delta$.
\end{theorem}
\begin{proof}
	Sei $x^*$ eine optimale Lösung von ($I$-\MIPI) gegeben.
	Für $\Delta=0$ ist $A$ unimodular und nach Lemma~\ref{lem:unimodular} ist $x^*\in\Z^n$ und optimal für ($\emptyset$-\MIPI).
	Wir betrachten also $\Delta\in\{1,2\}$.
	
	Es existiert ein $U\in\N$, sodass die beschränkte Menge  $$P:=\{x\in\R^n\mid Ax\leq b\} \cap \{x\in\R^n \mid \forall i\in\firstNumbers{m}: -U \leq x_i \leq U \}$$
	$x^*$ und eine optimale Lösung von ($J$-\MIPI) enthält.
	Setzt man $\tilde{A},\tilde{b}$ wie in Lemma~\ref{lem:bounded} ist $\rang(\tilde{A})=n$ und $\Delta:=\Delta(A)=\Delta(\tilde{A})$ und es gilt $P=\{ x\in\R^n \mid \tilde{A}x \leq \tilde{b} \}$.
	Es genügt nun, ein $y^*$ in $P$ zu finden, das für ($J$-\MIPI) optimal ist und $\norm{x^* - y^*}\leq\Delta$ erfüllt.
	
	\begin{description}
		\item[Fall 1:]  $I=\emptyset,J=\firstNumbers{n}, x^*$ ist Ecke von $P$.
		
		Ist $\Delta= 1$, so ist $x^*$ nach Lemma~\ref{lem:unimodular} ganzzahlig und $y^* := x^*$ ist die gewünschte Lösung mit $\norm{x^*-y^*}\leq \Delta - 1$.
		
		Sei also $\Delta=2$. Setze $Q:=\co{\{x\in\Z^n \mid A_\eq{x^*}x \leq b_\eq{x^*} \}}$.
		
		Für beliebiges $q\in Q\setminus\{x^*\}$ gilt:
		Es existiert $\lambda\in[0,1)$ mit $\tilde{q}:=\lambda q+(1-\lambda)x^*\in P$, da wir uns für jede Ungleichung, die $q$ noch nicht erfüllt, solange an $x^*$ annähern können bis sie erfüllt ist und diese Ungleichung für $x^*$ keine Gleichheit erfüllt.
		Ist $\lambda=0$, so ist $q\in P$ und $c\transpose q\leq c\transpose x^*$. Mit $c\transpose \tilde{q} \leq c\transpose x^*$ gilt sonst $c\transpose q=(c\transpose \tilde{q} - c\transpose(1-\lambda)x^*)/\lambda \leq c\transpose x^*$.
		Also ist $x\mapsto c\transpose x$ in $Q$ mit $c\transpose x^*$ nach oben beschränkt.
		
		
		Da $Q$ nicht-leer ist, existiert also eine Ecke $z\in\Z^n$ von $Q$, die $x\mapsto c\transpose x$ maximiert.
		Nach Lemma~\ref{lem:veselov}~(a) liegt $z$ auf einer Kante $E$ von $P$, die $x^*$ enthält.
		Sei $y^*\in\Z^n\cap E$ mit $\norm{x^* - y^*} := \min\{ \norm{x^*-y} \mid y\in\Z^n\cap E  \}$.
		Es existiert ein $\lambda\in [0,1]$ mit $y^* = \lambda z + (1-\lambda)x^*$ und mit $c\transpose x^* \leq c\transpose z$ gilt $c\transpose y^* \geq c\transpose z$.
		Aus der Optimalität von $z$ folgt die Optimalität von $y^*$ für ($\firstNumbers{n}$-\MIPI) und mit Lemma~\ref{lem:veselov}~(b) folgt aus der Wahl von $y^*$ die Abschätzung $\norm{x^*-y^*}\leq 1=\Delta-1$.

		\item[Fall 2:] $I=\emptyset$, $J=\firstNumbers{n}$.
		
		Die Menge aller optimalen Lösungen von ($\emptyset$-\MIPI) bildet eine Seitenfläche von $P$, nämlich $F:=\{x\in P\mid -c\transpose x \leq cx^* \}$.
		Sei $\tilde{z}$ eine optimale Lösung von ($\firstNumbers{n}$-\MIPI) und $\tilde{x}$ eine Ecke des Polytops $B:=\{ x\in F \mid \forall i \in \firstNumbers{n}: \Floor{x^*_i} \leq x_i \leq \Ceil{x^*_i} \}$.
		Dann folgt bereits $\norm{\tilde{x}-x^*}\leq 1$.
		
		Definiere die Polyeder
		$$
		\begin{array}{l}
		P_1:= \{ x\in\R^n \mid \forall i\in\firstNumbers{n}: (\tilde{y}_i\leq \Floor{x^*_i} = \tilde{x}_i \Rightarrow x_i \leq \Floor{x^*_i})\},\\
		P_2:= \{ x\in\R^n \mid \forall i\in\firstNumbers{n}: (\tilde{y}_i\geq \Floor{x^*_i} = \tilde{x}_i \Rightarrow x_i \geq \Floor{x^*_i})\}, \\
		P_3:= \{ x\in\R^n \mid \forall i\in\firstNumbers{n}: (\tilde{y}_i\leq \Ceil{x^*_i} = \tilde{x}_i \Rightarrow x_i \leq \Ceil{x^*_i})\},\\
		P_4:= \{ x\in\R^n \mid \forall i\in\firstNumbers{n}: (\tilde{y}_i\geq \Ceil{x^*_i} = \tilde{x}_i \Rightarrow x_i \geq \Ceil{x^*_i})\}
		\end{array}
		$$
		und das Polytop $\tilde{P}:=P\cap P_1 \cap P_2 \cap P_3 \cap P_4$.
		$\tilde{P}$ ist nicht-leer, da $\tilde{x}, \tilde{y}\in\tilde{P}$, und beschränkt, da $\tilde{P}\subseteq P$.
		Da jede an $\tilde{x}$ straffe Ungleichung von $B$ auch in $\tilde{P}$ vorkommt, ist $\tilde{x}$ auch eine optimale Ecke von $\tilde{P}$.
		
		$\tilde{P}$ kann wieder durch eine ganzzahlige Matrix $\tilde{A}$ beschrieben werden, die weiterhin $\rang(\tilde{A})=n$ und $\Delta(\tilde{A})=\Delta(A)$ erfüllt.
		Nun können wir Fall 1 anwenden und erhalten eine optimale Lösung $y^*$ in $\tilde{P}\cap\Z^n$, also auch eine optimale Lösung von ($\firstNumbers{n}$-\MIPI), mit $\norm{\tilde{x} -y^*}\leq \Delta -1$.
		Mit Dreiecksungleichung folgt $\norm{x^* - y^*}\leq \Delta$.
		
		\item[Fall 3:] $I=\firstNumbers{n},J=\emptyset,x^*$ ist Ecke von $R:=\co{\{ x\in\Z^n \mid Ax\leq b \}}$.
		
		Für $\Delta=1$ ist $x^*$ nach Lemma~\ref{lem:unimodular} auch eine optimale Lösung von ($\emptyset$-\MIPI).
		Wir betrachten den Fall $\Delta=2$.
		Nach Definition einer Ecke, existiert ein Vektor $d\in\R^n$ mit $\{ x\in\R^n \mid \forall \tilde{x}\in R: d\transpose x \geq d\transpose\tilde{x} \}=\{x^*\}$.
		Demnach ist $d\transpose x < d\transpose x^*$ für alle $x\in R\setminus\{x^*\}$.
		Für die Seitenfläche $F\subseteq P$, die alle optimalen Lösungen von ($\emptyset$-\MIPI) enthält, kann man nun ein $\lambda\geq0$ finden, das groß genug ist, sodass eine Ecke $z\in F$ existiert, die $x\mapsto (\lambda c+d)\transpose x$ über $P$ maximiert. \todo{Beweis Existenz $\lambda$?}
		
		Mit $\tilde{c}:=\lambda c+d$ gilt für jeden Punkt $x\in R\setminus\{x^*\}$:
		$$\tilde{c}\transpose x < \lambda c\transpose x + d\transpose x^*\leq \lambda c\transpose x^* + d\transpose x^* = \tilde{c}\transpose x^*.$$
		Setze $Q:=\co{\{x\in\Z^n \mid A_\eq{z}x \leq b_\eq{z} \}}$.
		Da $Q$ beschränkt \todo{?} ist, gibt es eine Ecke $v\in Q$, die nach Definition von $Q$ ganzzahlig ist und $x\mapsto\tilde{c}\transpose x$ über $Q$ maximiert.
		Nach Lemma~\ref{lem:veselov}~(a) liegt $v$ auf einer Kante $E$ von $P$, die $z$ enthält.
		Insbesondere ist also $Av\leq b$ und damit $v\in R$.
		Da auch $R \subseteq Q$, ist $v$ Maximierer von $x\mapsto \tilde{c}\transpose x$ über $R$ und damit ist $v=x^*$.
		
		Für den Fall $c\transpose x^*=c\transpose z$, so ist $x^*$ bereits optimal für ($\emptyset$-\MIPI).
		Sonst ist $x^* \neq z$ und, da $v=x^*$, liegt $x^*$ auf der Kante $E$ von $P$, die $z$ enthält.
		Auf der offenen Strecke zwischen $x^*$ und $z$ liegen keine ganzzahligen Punkte, weil $x\mapsto \tilde{c}\transpose x$ von $x^*$ über $R$ und von $z$ über $P\supseteq R$ maximiert und beide auf einer Kante von $P$ liegen.
		Also gilt nach Lemma~\ref{lem:veselov}~(b), dass $\norm{x^*-z}\leq 1 = \Delta -1 $
		
		\item[Fall 4:] $I=\firstNumbers{n},J=\emptyset.$
		
		Da $P$ beschränkt, ist auch $R:=\co{\{x\in\Z^n \mid Ax\leq b\}}$ beschränkt und, da $x^*\in R$, existiert eine Konvexkombination $x^* = \sum_{i=1}^t \lambda_i v^i$ mit $v^1,\dots,v^t$ Ecken von $R$ und $\lambda_1,\dots,\lambda_t>0$ und $\sum_{i=1}^t \lambda_i=1$.
		Angenommen für ein $k\in\firstNumbers{t}$ ist $v^k$ nicht optimal für ($\firstNumbers{n}$-\MIPI).
		Dann ist $c\transpose x^*=\sum_{i=1}^t \lambda_i c\transpose v^i \leq (1-\lambda_k)c\transpose x^* + \lambda_k c\transpose v^k < c\transpose x^*$.
		Also ist $v^i$ optimal für ($\firstNumbers{n}$-\MIPI) und nach Fall 3 existiert $z^i\in\R^n$ optimal für ($\emptyset$-\MIPI) mit $\norm{v^i-z^i}\leq \Delta$ für alle $i\in\firstNumbers{t}$.
		Die Konvexkombination $y^*:=\sum_{i=1}^t \lambda_i z^i$ ist ebenfalls optimal für ($\emptyset$-\MIPI) und
		$$\norm{x^*-y^*}=\norm{\sum_{i=1}^t \lambda_i (v^i-z^i)}\leq \sum_{i=1}^t \lambda_i \norm{v^i-z^i}=\Delta.$$
	\end{description}
\end{proof}