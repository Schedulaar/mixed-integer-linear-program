\section{Lineare Abhängigkeit von $\Delta$}\label{sec:linear}

Bisher wurden nur Abschätzungen mit $b\in\R^m$ betrachtet.
Schrijver hat in~\cite[Kapitel~17.2]{Schrijver1986} ein Beispiel angeführt, das $n\Delta$ als beste Abschätzung von optimalen Lösungen von ($\emptyset$-\MIPR) und ($\firstNumbers{n}$-\MIPR) besitzt.
Schränkt man das Problem mit $b\in\Z^m$ ein, kann man mit folgendem Beispiel erkennen, dass der Abstand optimaler Lösungen zumindest linear von $\Delta$ abhängt:
\begin{example}
	Für $\delta\in\N$ seien
	$$A:=
	\begin{pmatrix}
	-\delta & 0  \\
	\delta  & -1
	\end{pmatrix},\quad
	b:=\begin{pmatrix} -1 \\ 0 \end{pmatrix},\quad
	c\transpose:=(0, -1).
	$$
	Es wird also die zweite Komponente minimiert unter der Nebenbedingung $Ax\leq b$, also $\delta x_1-x_2\leq0\Leftrightarrow\delta x_1\leq x_2$ und $\delta x_1\geq 1$.
	Diese können zusammengefasst werden in $1\leq\delta x_1\leq x_2$.
	Es gilt $\Delta=\delta$ und die optimale Lösung von ($\emptyset$-\MIPI) und (\{2\}-\MIPI) ist $x^*=(1/\delta,1)$.
	Die optimale Lösung von ($\{1\}$-\MIPI) und ($\{1, 2\}$-\MIPI) ist jedoch $y^*=(1,\delta)$.
	Entsprechend ist der Abstand $\norm{x^*-y^*}=\delta-1=\Omega(\Delta)$.
\end{example}

Im restlichen Abschnitt wird das Problem weiter eingeschränkt, um in diesen Fällen die Vermutung~\ref{con:delta} zeigen zu können.
Dabei werden keine echt gemischt-ganzzahligen Programme mehr zugelassen, sondern nur noch Indexmengen in $\{\emptyset, \firstNumbers{n} \}$.

\begin{notation}
	Seien $A\in\R^{m\times n}$, $b\in\R^m$ und $x^*\in\R^n$ gegeben.
	Im Kontext des Polyeders $\{ x\in\R^m \mid Ax\leq b \}$ bezeichnen
	$A_\eq{x^*}$ und $b_\eq{x^*}$ diejenigen Zeilen aus $A$ und $b$, bei denen die zugehörigen Ungleichungen für $x^*$ straff sind.
	
	Für eine Menge $M\subseteq \R^n$ bezeichne $\co{M}$ ihre konvexe Hülle.
\end{notation}

Das folgende Lemma aus~\cite[Theorem 2 und Beweis]{VESELOV2009220} wird nun benutzt, um für einige Situationen die Vermutung~\ref{con:delta} zu bestätigen:
\begin{lemma}[Veselov-Chirkov, 2009]\label{lem:veselov}
	Seien eine ganzzahlige Matrix $A\in\Z^{m\times n}$, $b\in\Z^m$ mit $\rang(A)=n$ gegeben und der Betrag jeder Determinante einer $n\times n$ Teilmatrix von $A$ sei kleinergleich 2.
	
	Sei $z$ eine Ecke von $P:=\{x\in\R^n\mid Ax\leq b \}$ und sei $Q:=\co{\{x\in\Z^n \mid A_\eq{z}x \leq b_\eq{z} \}}$.
	Dann gelten:
	\begin{enumerate}[(a)]
		\item Jede Ecke von $Q$ liegt auf einer Kante von $P$, die $z$ enthält.
		\item Jede Kante von $P$, die $z$ und einen ganzzahligen Punkt enthält, enthält auch einen ganzzahligen Punkt $y$ mit $\norm{z -y}\leq 1$.
	\end{enumerate}
\end{lemma}
\begin{lemma}\label{lem:same-delta}
	Seien $A\in\R^{m\times n}$ mit $\Delta(A)\geq 1$ gegeben und $e\in\R^n$ ein Vektor, der in einem Eintrag den Wert $1$ oder $-1$ hat und in allen anderen den Wert $0$.
	Erhält man die Matrix $\tilde{A}$ durch Einfügen von $e$ als Zeile oder Spalte bei $A$, so bleibt $\Delta(\tilde{A})=\Delta(A)$ erhalten.
\end{lemma}
\begin{proof}
	Da jede Untermatrix von $A$ auch eine Untermatrix von $\tilde{A}$ ist, folgt die Ungleichung $\Delta(A)\leq\Delta(\tilde{A})$.
	
	Sei nun $M$ eine quadratische Untermatrix von $\tilde{A}$, die keine Untermatrix von $A$ ist.
	Ohne Einschränkung kann davon ausgegangen werden, dass $e$ als letzte Zeile an $A$ angefügt worden ist, da sich der Betrag der Determinante unter Transponieren und Zeilentausch nicht verändert.
	Die ersten Zeilen von $M$ bilden also eine Untermatrix von $A$, die letzte Zeile $\tilde{e}$ ist ein Untervektor von $e$.
	
	Ist $\tilde{e}=0$, so ist $\det(M)=0$.
	Sonst lässt sich $M$ mit elementaren Zeilen- und Spaltentransformationen in die Form $$\tilde{M}:= \begin{pmatrix}
		B & 0 \\
		0 & 1
	\end{pmatrix}$$ bringen, wobei $B$ eine Untermatrix von $A$ ist und $\betrag{\det(M)}=\betrag{\det(\tilde{M})}$ gilt.
	Ist $B$ leer, so ist $\betrag{\det(M)}=1$, sonst gilt $\betrag{\det(M)}=\betrag{\det(B)}\leq\Delta(A)$.
\end{proof}

Als Nächstes wird gezeigt, dass für Probleme in Standardform mit einer total unimodularen Matrix alle Ecken ganzzahlig sind.
Eine \emph{total unimodulare Matrix} ist dabei eine Matrix, bei der die Determinante jeder quadratischen Untermatrix in $\{-1,0,1\}$ liegt.
Da jede total unimodulare Matrix bereits ganzzahlig ist, kann die Eigenschaft auch durch $\Delta(A)\leq 1$ für ganzzahlige Matrizen $A$ charakterisiert werden.

\begin{lemma}\label{lem:unimodular}
	Seien $A\in\Z^{m\times n},b\in\Z^m$ gegeben mit $\Delta(A)\leq 1$.
	Dann ist jede Ecke von $\{x\in\R^n \mid Ax\leq b \}$ ganzzahlig.
\end{lemma}
\begin{proof}
	Für eine Ecke $x$ gilt $\rang(A_\eq{x})=n$.
	Nach dem Basisauswahlsatz können, falls nötig, Zeilen von $A_\eq{x}$ und $b_\eq{x}$ entfernt werden, um eine reguläre Teilmatrix $\tilde{A}$ von $A_\eq{x}$ mit zugehörigem $\tilde{b}$ zu erhalten.
	Nun kann $\tilde{A}x=\tilde{b}$ mit der Cramerschen Regel eindeutig gelöst werden.
	Dabei ist $x_i=\det(\tilde{A}_i)/\det(\tilde{A})$, wobei $\tilde{A}_i$ aus $\tilde{A}$ durch Ersetzen der $i$-ten Spalte mit $\tilde{b}$ entsteht.
	Da $\det(\tilde{A})\in\{-1,1\}$ gilt und $\tilde{A_i}$ ganzzahlig ist, ist auch $x_i$ ganzzahlig.
\end{proof}
\begin{lemma}\label{lem:q-upper-bound}
	Seien $A\in\R^{m\times n}$, $b\in\R^m$ und $c\in\R^n$ sowie $x^*\in P:=\{ x\in\R^n \mid Ax\leq b \}$ gegeben.
	 $c\transpose x^*$ ist obere Schranke von $x\mapsto c\transpose x$ über $Q:=\co{\{x\in\Z^n \mid A_\eq{x^*}x \leq b_\eq{x^*} \}}$, falls $x\mapsto c\transpose x$  von $x^*$ über $P$ maximiert wird.
\end{lemma}
\begin{proof}
	Für beliebiges $q\in Q\setminus\{x^*\}$ gilt:
	Es existiert $\lambda\in[0,1)$ mit $\tilde{q}:=\lambda q+(1-\lambda)x^*\in P$, da man sich für jede Ungleichung, die $q$ noch nicht erfüllt, solange an $x^*$ annähern kann bis sie erfüllt ist, und da diese Ungleichung für $x^*$ nicht straff ist.
	Ist $\lambda=0$, so liegt $q$ in~$P$ und $c\transpose q\leq c\transpose x^*$. Mit $c\transpose \tilde{q} \leq c\transpose x^*$ gilt sonst $c\transpose q=(c\transpose \tilde{q} - c\transpose(1-\lambda)x^*)/\lambda \leq c\transpose x^*$.
	Also ist $x\mapsto c\transpose x$ in $Q$ mit $c\transpose x^*$ nach oben beschränkt.
\end{proof}

Zunächst wird die gewünschte Abschätzung für $J=\firstNumbers{n}$ und $I=\emptyset$ gezeigt:

\begin{lemma}\label{lem:i-n-j-e}
	Seien $\Delta\in\{1,2\}$, $P:=\{x\in\R^n \mid Ax\leq b \}$ beschränkt und es existiere eine optimale Lösung von ($\firstNumbers{n}$-\MIPI).
	Dann existiert für jede optimale Lösung $x^*$ von ($\emptyset$-\MIPI) eine optimale Lösung $y^*$ von ($\firstNumbers{n}$-\MIPI) mit $\norm{x^*-y^*}\leq \Delta$.
\end{lemma}
\begin{proof}
	\begin{description}
		\item[Spezialfall:] $x^*$ ist Ecke von $P$.
		
		Ist $\Delta= 1$, so ist $x^*$ nach Lemma~\ref{lem:unimodular} ganzzahlig und $y^* := x^*$ ist die gewünschte Lösung mit $\norm{x^*-y^*}= 0\leq \Delta-1$. Sei also $\Delta=2$.
		
		Setze $Q:=\co{\{x\in\Z^n \mid A_\eq{x^*}x \leq b_\eq{x^*} \}}$.
		Da $Q$ nicht-leer ist und $x\mapsto c\transpose x$ nach Lemma~\ref{lem:q-upper-bound} mit $c\transpose x^*$ über $Q$ nach oben beschränkt ist, existiert also eine Ecke $z\in\Z^n$ von $Q$, die $x\mapsto c\transpose x$ maximiert.
		Nach Lemma~\ref{lem:veselov}~(a) liegt $z$ auf einer Kante $E$ von $P$, die $x^*$ enthält.
		
		Sei $y^*\in\Z^n\cap E$ mit $\norm{x^* - y^*} := \min\{ \norm{x^*-y} \mid y\in\Z^n\cap E  \}$.
		Da $x^*$ Ecke der gemeinsamen Kante $E$ ist, existiert ein $\lambda\in [0,1]$ mit $y^* = \lambda z + (1-\lambda)x^*$ und mit $c\transpose x^* \leq c\transpose z$ gilt $c\transpose y^* \geq c\transpose z$.
		Aus der Optimalität von $z$ folgt die Optimalität von $y^*$ für ($\firstNumbers{n}$-\MIPI) und mit Lemma~\ref{lem:veselov}~(b) folgt aus der Wahl von $y^*$ die Abschätzung $\norm{x^*-y^*}\leq 1=\Delta-1$.
		
		\item[Allgemeiner Fall.] ~
		
		Die Menge aller optimalen Lösungen von ($\emptyset$-\MIPI) bildet eine Seitenfläche von $P$, nämlich $F:=\{x\in P\mid c\transpose x \geq c\transpose x^* \}$.
		Sei $\tilde{z}$ eine optimale Lösung von ($\firstNumbers{n}$-\MIPI) und $\tilde{x}$ eine Ecke des Polytops $B:=\{ x\in F \mid \forall i \in \firstNumbers{n}: \Floor{x^*_i} \leq x_i \leq \Ceil{x^*_i} \}$.
		Dann folgt bereits $\norm{\tilde{x}-x^*}\leq 1$.
		
		Definiere die Polyeder
		$$
		\begin{array}{l}
		P_1:= \{ x\in\R^n \mid~ \tilde{y}_i\leq \Floor{x^*_i} = \tilde{x}_i ~\Rightarrow~ x_i \leq \Floor{x^*_i}\qquad\text{für alle}~ i\in\firstNumbers{n}\},\\
		P_2:= \{ x\in\R^n \mid ~ \tilde{y}_i\geq \Floor{x^*_i} = \tilde{x}_i ~\Rightarrow~ x_i \geq \Floor{x^*_i}\qquad\text{für alle}~ i\in\firstNumbers{n}\}, \\
		P_3:= \{ x\in\R^n \mid ~\tilde{y}_i\leq \Ceil{x^*_i} = \tilde{x}_i ~\Rightarrow~ x_i \leq \Ceil{x^*_i}\qquad\text{für alle}~ i\in\firstNumbers{n}\},\\
		P_4:= \{ x\in\R^n \mid ~\tilde{y}_i\geq \Ceil{x^*_i} = \tilde{x}_i ~\Rightarrow~ x_i \geq \Ceil{x^*_i}\qquad\text{für alle}~ i\in\firstNumbers{n}\}
		\end{array}
		$$
		und das Polytop $\tilde{P}:=P\cap P_1 \cap P_2 \cap P_3 \cap P_4$.
		$\tilde{P}$ ist nicht-leer, da $\tilde{x}, \tilde{y}\in\tilde{P}$, und beschränkt, da $\tilde{P}\subseteq P$.
		Da jede an $\tilde{x}$ straffe Ungleichung von $B$ auch in $\tilde{P}$ vorkommt, ist $\tilde{x}$ auch eine optimale Ecke von $\tilde{P}$.
		
		$\tilde{P}$ kann wieder durch eine ganzzahlige Matrix $\tilde{A}$  und einen ganzzahligen Vektor $\tilde{b}$ beschrieben werden, wobei $\tilde{A}$ nach Lemma~\ref{lem:same-delta} weiterhin $\rang(\tilde{A})=n$ und $\Delta(\tilde{A})=\Delta(A)$ erfüllt, da nur einige Zeilen mit einer einzelnen $\pm 1$ in einer der Spalten hinzukommen.
		Nun kann der Spezialfall angewandt werden und man erhält eine optimale Lösung $y^*\in\Z^n$ in $\tilde{P}$, also auch eine optimale Lösung von ($\firstNumbers{n}$-\MIPI), mit $\norm{\tilde{x} -y^*}\leq \Delta -1$.
		Mittels Dreiecksungleichung folgt $\norm{x^* - y^*}\leq \Delta$.
	\end{description}
\end{proof}

\begin{lemma}\label{lem:i-e-j-n}
	Seien $\Delta\in\{1,2\}$, $P:=\{x\in\R^n \mid Ax\leq b \}$ beschränkt und es existiere eine optimale Lösung von ($\emptyset$-\MIPI).
	Dann existiert für jede optimale Lösung $x^*$ von ($\firstNumbers{n}$-\MIPI) eine optimale Lösung $y^*$ von ($\emptyset$-\MIPI) mit $\norm{x^*-y^*}\leq \Delta$.
\end{lemma}
\begin{proof}
	Für $\Delta=1$ ist nach Lemma~\ref{lem:unimodular} jede Ecke ganzzahlig, und da eine optimale Ecke existiert, ist $x^*$ auch eine optimale Lösung von ($\emptyset$-\MIPI).
		\begin{description}
		
		
		\item[Spezialfall:] $x^*$ ist Ecke von $R:=\co{\{ x\in\Z^n \mid Ax\leq b \}}$.
		
		Nach Definition einer Ecke, existiert ein Vektor $d\in\R^n$ mit $$\{ x\in\R^n \mid \forall \tilde{x}\in R: d\transpose x \geq d\transpose\tilde{x} \}=\{x^*\}.$$
		Demnach ist $x^*$ der einzige Maximierer von $x\mapsto d\transpose x$ über $R$.
		
		Die Seitenfläche aller optimalen Lösungen von ($\emptyset$-\MIPI) wird mit $F$ bezeichnet.
		Ist $x^*\in F$, so ist $x^*$ bereits optimal für ($\emptyset$-\MIPI).
		Sonst sei $y^*\in F$ eine Ecke, die $x\mapsto d\transpose x$ über $F$ maximiert.
		Setze $\lambda\geq 0$ so groß, dass $\lambda\geq d\transpose (v-y^*)/c\transpose (y^*-v)$ für alle Ecken $v\in P\setminus F$ von $P$ gilt.
		Mit $\tilde{c}:= \lambda c + d$ maximiert $y^*$ die Funktion $x\mapsto \tilde{c}\transpose x$ über $P$, denn für alle Ecken $v$ in  P gilt: Ist $v\in F$, dann ist $\tilde{c}\transpose v = \lambda c\transpose y^* + \lambda d\transpose v\leq \tilde{c}\transpose y^*$, und sonst ist 
		$\tilde{c}\transpose (y^*-v)\geq (d\transpose (v-y^*)/c\transpose(y^*-v))
		c\transpose (y^*-v) + d\transpose (y^*-v)=0$
		
		Setze $Q:=\co{\{x\in\Z^n \mid A_\eq{y^*}x \leq b_\eq{y^*} \}}$.
		Nach Lemma~\ref{lem:q-upper-bound} ist $x\mapsto \tilde{c}\transpose x$ mit $\tilde{c}\transpose y^*$ in $Q$ nach oben beschränkt und, da $Q$ nicht-leer ist, gibt es eine Ecke $v\in\Z^n$ von $Q$, die $x\mapsto\tilde{c}\transpose x$ über $Q$ maximiert.
		Nach Lemma~\ref{lem:veselov}~(a) liegt $v$ auf einer Kante $E$ von $P$, die $y^*$ enthält.
		Insbesondere ist also $Av\leq b$ und damit $v\in R$.
		Da auch $R \subseteq Q$, ist $v$ Maximierer von $x\mapsto \tilde{c}\transpose x$ über $R$ und damit ist $v=x^*$.
		
		Da $v=x^*$, liegt $x^*$ auf der Kante $E$ von $P$, die $y^*$ enthält.
		Auf der offenen Strecke zwischen $x^*$ und $y^*$ liegen keine ganzzahligen Punkte, weil $x\mapsto \tilde{c}\transpose x$ von $x^*$ über $R$ und von $y^*$ über $P\supseteq R$ maximiert wird und beide auf einer Kante von $P$ liegen.
		Also gilt nach Lemma~\ref{lem:veselov}~(b), dass $\norm{x^*-y^*}\leq 1 = \Delta -1 $
		
		\item[Allgemeiner Fall.] ~
		
		Da $P$ beschränkt, ist auch $R:=\co{\{x\in\Z^n \mid Ax\leq b\}}$ beschränkt und, da $x^*\in R$, existiert eine Konvexkombination $x^* = \sum_{i=1}^t \lambda_i v^i$ mit $v^1,\dots,v^t$ Ecken von $R$ und $\lambda_1,\dots,\lambda_t>0$ und $\sum_{i=1}^t \lambda_i=1$.
		
		Angenommen für ein $k\in\firstNumbers{t}$ ist $v^k$ nicht optimal für ($\firstNumbers{n}$-\MIPI).
		Dann ist $c\transpose x^*=\sum_{i=1}^t \lambda_i c\transpose v^i \leq (1-\lambda_k)c\transpose x^* + \lambda_k c\transpose v^k < c\transpose x^*$.
		Also ist $v^i$ optimal für ($\firstNumbers{n}$-\MIPI) und nach dem Spezialfall existiert $z^i\in\R^n$ optimal für ($\emptyset$-\MIPI) mit $\norm{v^i-z^i}\leq \Delta$ für alle $i\in\firstNumbers{t}$.
		Die Konvexkombination $y^*:=\sum_{i=1}^t \lambda_i z^i$ ist ebenfalls optimal für ($\emptyset$-\MIPI) und
		$$\norm{x^*-y^*}=\norm{\sum_{i=1}^t \lambda_i (v^i-z^i)}\leq \sum_{i=1}^t \lambda_i \norm{v^i-z^i}=\Delta.$$
	\end{description}
\end{proof}
Die Ergebnisse der letzten beiden Lemmata werden in folgendem Theorem festgehalten:
\begin{theorem}
		Seien $\Delta\leq 2$ und $I,J\in\{\emptyset,\firstNumbers{n}\}$, sodass eine optimale Lösung von ($J$-\MIPI) existiert.
		Für jede optimale Lösung $x^*$ von ($I$-\MIPI) existiert eine optimale Lösung $y^*$ von ($J$-\MIPI) mit $\norm{x^*-y^*}\leq \Delta$.
\end{theorem}
\begin{proof}
	Sei $x^*$ eine optimale Lösung von ($I$-\MIPI) gegeben.
	Für $\Delta=0$ ist $A=0$ und damit jedes $x\in\R^n$ optimale Lösung.
	Man betrachte also $\Delta\in\{1,2\}$.
	
	Es existiert ein $U\in\N$, sodass die beschränkte Menge  $$P:=\{x\in\R^n\mid Ax\leq b\} \cap \{x\in\R^n \mid \forall i\in\firstNumbers{m}: -U \leq x_i \leq U \}$$
	$x^*$ und eine optimale Lösung von ($J$-\MIPI) enthält.
	Setzt man
	$$\tilde{A}:=\begin{pmatrix} A \\ -\one \\ \one \end{pmatrix},\qquad\tilde{b}:=\begin{pmatrix} b \\ U \\ U \end{pmatrix},$$
	gilt $\rang(\tilde{A})=n$ und nach Lemma~\ref{lem:same-delta} ist $\Delta(A)=\Delta(\tilde{A})$.
	$P$ kann nun in der Form $P=\{ x\in\R^n \mid \tilde{A}x \leq \tilde{b} \}$ dargestellt werden.
	Es genügt, ein $y^*$ in $P$ zu finden, das für ($J$-\MIPI) optimal ist und $\norm{x^* - y^*}\leq\Delta$ erfüllt.
	
	Mit Lemma~\ref{lem:i-n-j-e} und Lemma~\ref{lem:i-e-j-n} folgt die Behauptung.
\end{proof}