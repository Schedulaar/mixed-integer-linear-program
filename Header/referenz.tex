
% Literatur-Bibliothek
\bibliographystyle{alphadin}               % deutscher Bibliotheksstil

% Interaktive Referenzen, und PDF-Keys
\usepackage{xr-hyper}  
\usepackage[pagebackref,                % R\"uckreferenz im Literaturverzeichnis
           ps2pdf,                      % Treiber f\"ur ps zu pdf ; f\"ur direkt nach pdf: pdftex
           ]{hyperref}

% Erweiterte Einstellungen zu hyperref

\hypersetup{
        breaklinks=true,        % zu lange Links unterbrechen
        colorlinks=true,        % F\"arben von Referenzen
        citecolor=black,        % Farbe der Zitate
        linkcolor=black,        % Farbe der Links
        extension=pdf,          % Externe Dokumente k\"onnen eingebunden werden.
        ngerman,		
	pdfview=FitH,
	pdfstartview=FitH,		
	bookmarksnumbered=true, % Anzeige der Abschnittsnummern	% pdf-Titel
	pdfauthor={\autor}% pdf-Autor
}

% Namen f\"ur Referenzen 

\newcommand{\ownautorefnames}{
  \renewcommand{\sectionautorefname}{Kapitel}
  \renewcommand{\subsectionautorefname}{Unterkapitel}
  \renewcommand{\subsubsectionautorefname}{\subsectionautorefname}
  \renewcommand{\appendixautorefname}{Anhang}
  \renewcommand{\figureautorefname}{Abbildung}
}

% R\"uckreferenzentext zur Literatur
\def\bibandname{und}%
\renewcommand*{\backref}[1]{}
\renewcommand*{\backrefalt}[4]{%
\ifcase #1 %
 (Nicht zitiert, also Erg\"anzungsliteratur.)%
\or
 (Zitiert auf Seite #2.)%
\else
 (Zitiert auf den Seiten #2.)%
\fi
}
\renewcommand{\backreftwosep}{ und~} % seperate 2 pages
\renewcommand{\backreflastsep}{ und~} % seperate last of longer 

