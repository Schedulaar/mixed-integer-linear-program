%-------------------------------------------------------------------------------
% Hilfreiche Befehle
%-------------------------------------------------------------------------------
\newcommand{\betrag}[1]{\lvert #1 \rvert}	        % Betrag
\providecommand*{\Lfloor}{\left\lfloor}                 % gro\ss{}es Abrunden
\providecommand*{\Rfloor}{\right\rfloor}                % gro\ss{}es Abrunden
\providecommand*{\Floor}[1]{\Lfloor #1 \Rfloor}         % gro\ss{}es ganzes Abrunden
\providecommand*{\Ceil}[1]{\left\lceil #1 \right\rceil} % gro\ss{}es ganzes Aufrunden

\newcommand{\Z}{\mathbb{Z}}
\newcommand{\N}{\mathbb{N}}
\newcommand{\R}{\mathbb{R}}
\newcommand{\Q}{\mathbb{Q}}
\newcommand{\firstNumbers}[1]{[#1]}
\newcommand{\transpose}{^\intercal}
\newcommand{\subjectTo}{\textbf{s.t.}}
\newcommand{\MIPR}{MIP\textsuperscript{*}}
\newcommand{\MIPI}{MIP}
\newcommand{\oBdA}{oBdA.}
\newcommand{\rang}{\operatorname{rang}}
\newcommand{\norm}[1]{\left\lVert#1\right\rVert_\infty}
\newcommand{\zero}{0}
\newcommand{\todo}[1]{{\color{red}{\emph{TODO: }}#1}}
\newcommand{\one}{\mathbbm{1}}
\newcommand{\eq}[1]{{\operatorname{eq}(#1)}}
\newcommand{\co}[1]{\operatorname{co}(#1)}