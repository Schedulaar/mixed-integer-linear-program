\usepackage{array}		% erweiterte Tabellen

% Schriftzeichen, Format
\usepackage{latexsym}		% Latex-Symbole
\usepackage[english, german, ngerman]{babel}	% Mehrsprachenumgebung

% Layout
\usepackage{geometry}                    % Seitenränder
\geometry{a4paper, top=30mm, bottom=30mm, left=30mm, right=30mm}
\addtolength{\footskip}{-0.5cm}          % Seitenzahlen höher setzen
\usepackage{xcolor}                      % Farben
\usepackage{mathabx}


% Tabellen und Listen
\usepackage{float}		        % Gleitobjekte 
\usepackage[flushright]{paralist}       % Bessere Behandlung der Auflistungen

% Bilder
\usepackage[final]{graphicx}            % Graphiken einbinden

\usepackage{caption}                    % Beschriftungen
\usepackage{subcaption}                 % Beschriftungen f\"ur Unterteilung

\usepackage{pst-all}                    % Zeichnungen in Latex (kein pdflatex)
\usepackage{pstricks-add}               % zus\"atzliches von pstricks
\usepackage{pst-3dplot}                 % dreidimensionale Zeichnungen
\usepackage{pst-eucl}                   % euklidisches Paket

\numberwithin{figure}{section}	% Abbildungsnummern in Section

\newcommand{\Z}{\mathbb{Z}}
\newcommand{\N}{\mathbb{N}}
\newcommand{\R}{\mathbb{R}}
\newcommand{\Q}{\mathbb{Q}}
\newcommand{\firstNumbers}[1]{[#1]}
\newcommand{\transpose}{^T}
\newcommand{\subjectTo}{\textbf{s.t.}}
\newcommand{\MIP}{MIP}
\newcommand{\oBdA}{oBdA.}
\newcommand{\norm}[1]{\left\lVert#1\right\rVert_\infty}

\newcommand{\todo}[1]{{\color{red}{\emph{TODO: }}#1}}
