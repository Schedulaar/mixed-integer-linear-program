Ein \emph{gemischt-ganzzahliges Programm} (mixed integer program, MIP) ist ein lineares Programm, das von einer ganzzahligen Matrix $A\in\Z^{n\times m}$, einer ganzzahligen rechten Seite $b\in\Z^m$, $c\in\R^n$ und einer Indexmenge $I\subseteq\firstNumbers{n}:=\{1,\dots,n\}$ erzeugt wird, und von der folgenden Form ist:
\begin{equation}\tag{$I$-\MIPI}
\begin{array}{lc}
&\max c\transpose x \\
\subjectTo &Ax\leq b\\
&\forall i\in I: x_i\in\Z
\end{array}.
\end{equation}

Gesucht ist für $I,J\subseteq\firstNumbers{n}$ eine möglichst kleine Schranke, die den Abstand zwischen jeder optimalen Lösung $x^*$ von ($I$-\MIPI) und einer zu $x^*$ nähesten optimalen Lösung $y^*$ von ($J$-\MIPI) beschränkt.
Definiert man
$$\Delta:=\Delta(A):=\max\{\betrag{ \det(Q)} \mid Q \text{ quadratische Untermatrix von } A \},$$
so erhält man die in \cite{Paat2018} formulierte Vermutung:

\begin{conjecture}\label{con:delta}
	Es gibt eine Funktion $f: \N\rightarrow\R$, sodass für alle $I,J\subseteq\firstNumbers{n}$, unter denen ($J$-\MIPI) eine optimale Lösung besitzt, gilt:
	Besitzt ($I$-\MIPI) eine optimale Lösung $x^*$, so existiert eine optimale 
	Lösung $y^*$ von ($J$-\MIPI) mit $\norm{x^* - y^*}\leq f(\Delta)$.
\end{conjecture}

Eine Abschätzung, die zusätzlich von der Dimension $n$ abhängt, lieferte bereits Cook in~\cite[Theorem 1 und Bemerkung 1]{Cook1986}:

\begin{theorem}[Cook et al., 1986]\label{thm:cook}
	Seien $I, J\subseteq\firstNumbers{n}$, sodass ($J$-\MIPI) eine optimale Lösung hat und entweder $I=\emptyset$ oder $J=\emptyset$ gilt.
	Dann existiert für jede optimale Lösung $x^*$ von ($I$-\MIPI) eine optimale Lösung $y^*$ von ($J$-\MIPI) mit $\norm{x^*-y^*}\leq n\Delta$.
\end{theorem}

Daraus kann leicht eine Abschätzung für allgemeine Indexmengen $I,J$ mit der oberen Schranke $2n\Delta$ gefolgert werden.
Diese soll nun verstärkt werden, indem die Schranke auf $\betrag{I\cup J}\Delta$ reduziert wird, wobei $\betrag{I\cup J}$ die Anzahl ganzzahliger Variablen ist.

Dabei lässt sich aus der Theorie endlicher abelscher Gruppen ein wichtiges Hilfslemma beziehen, das auf einem Theorem von Olson in~\cite{Olson1969} beruht.
Die sogenannte \emph{Davenport-Konstante $D(G)$} einer endlicher abelschen Gruppe $G$ bezeichnet dabei die kleinste natürliche Zahl $k$, für die gilt:
$$
\forall g^1,\dots,g^k \in G~~\exists I\subseteq\firstNumbers{k}\colon I\neq\emptyset \wedge \sum_{i\in I}g^i=0 .
$$
\begin{theorem}[Olson, 1969]
	Für eine Primzahl $p$ ist $D(\Z^d/p\Z^d)=1+dp-p$.
\end{theorem}
\begin{corollary}
	Seien $p\in\N$ eine Primzahl, $d\in\N$ und $f^1,\dots,f^r\in\Z^d$ mit $r\geq 1+dp-d$ gegeben.
	Dann existiert eine nicht-leere Menge $I\subseteq\firstNumbers{r}$ mit $\sum_{i\in I}f^i\in p\Z^d$.
\end{corollary}

Mit Hilfe dieses Korollars kann eine Reihe weitere Aussagen gezeigt werden, die für den Beweis der Abschätzung notwendig sind:

\begin{lemma}\label{lem:olson}
	Seien $d,k\in\N, u^1,\dots, u^k\in\Z^d$ und $\alpha_1,\dots,\alpha_k\geq0$ mit $\sum_{i=1}^k \alpha_i\geq d$.
	Dann existiert $\beta\in\bigtimes_{i=1}^k[0,\alpha_i]$ mit $\beta\neq0$ und $\sum_{i=1}^k\beta_i u^i \in\Z^d$.
\end{lemma}

\begin{lemma}\label{lem:maxgamma}
	Seien $u^i\in\Z^d$, $\lambda_i\geq0$ für $i\in\firstNumbers{k}$ gegeben.
	Dann existiert $\gamma\in\bigtimes_{i=1}^k [0,\lambda_i]$ mit $\sum_{i=1}^k \gamma_i u^i\in\Z^d$ und  $\sum_{i=1}^k(\lambda_i-\gamma_i)<d$.
\end{lemma}

Außerdem spielt das folgende Lemma über die Darstellung von polyedrischen Kegeln eine wichtige Rolle, das zum Beispiel in~\cite{bibid}:

\begin{lemma}
	Sei $A\in\Z^{m\times n}$ gegeben.
	Der Polyeder $C:=\{ x\in\R^n \mid A x\leq 0 \}$ besitzt die Darstellung $C=\{\sum_{i=1}^k\lambda_i v^i \mid \forall i\in\firstNumbers{k} \colon\lambda_i\geq0 \}$ mit $v^i\in\Z^n$ und $\norm{v^i}\leq\Delta(A)$ für $i\in\firstNumbers{k}$.
\end{lemma}